\section{Introduction}
\label{sec:intro}

Relational learning refers to the inference of rules that operate in terms of
relationships between objects, independent of how the objects may be represented. Examples of common relations are
``less than'' applied to natural numbers, ``same color as'' applied to visual objects, and ``friend of'' applied to
social relationships. Such relations are important for basic computational functions.  For example, consider the task
of sorting objects. A standard 52-card deck of playing cards has 13 ranks in each of four suits: clubs, diamonds,
hearts, and spades. The cards might be sorted using the relation that takes the suit as the primary attribute,
and the rank as the secondary attribute; this would be the typical
way of ordering cards in many games. If a sorting algorithm is learned that depends 
only on the relations between objects, it can in principle be applied in a new domain with little 
or no modification. 

Reasoning in terms of relations and analogies is a hallmark of human intelligence 
\citep{snow,holyoak}.
% I'd prefer to keep it as is, since the sorting theme is important in the paper.
% JDC: I'D BE INCLINED TO REPLACE THIS:
Indeed, the Wisconsin card sorting task \citep{berg} has been used for decades as an indicator of decision making
function in prefrontal cortex \citep{monchi}.
% WITH THIS:
%-----------
% Indeed, the Ravens Progressive Matrices [CITE] has been used for decades as an indicator of abstract reasoning and
% general intelligence.
% ----------
% SINCE: A) WCST IS NOT A VERY SOPHISTICATED TEST AND, AS NOTED, HAS BEEN USED MORE FOR EVALUATING PFC FUNCTION THAN
% TESTING REASONING AND ABSTRACTION ABILITY (IN FACT, IT DOES THIS ONLY IN A RUDIMENTARY WAY, AND IS SENSITIVE TO
% OTHER FACTORS, SUCH AS REINFORCEMENT LEARNING, ETC.).  ALSO, RAVENS IS NOW PRETTY FAMILIAR TO THE ML COMMUNITY.
% CITE:  Raven, J. (2003). Raven progressive matrices (pp. 223-237). Springer US. [TAYLOR, FEEL FREE TO REPLACE IF
% YOU AGREE WITH THE ABOVE BUT HAVE A PREFERRED CITE]
Recognizing the importance of this capability,
which is largely separate from function approximation for sensory tasks such as image and audio processing, machine
learning research has explored several novel frameworks for relational learning \citep{TEM, NTM,episodicControl,esbn,mondal23learned,
battaglia,barrett:2018,santoro1}.

In this paper we propose a framework that casts relational learning in terms of transformers. 
The success of transformers lies in combining the function approximation capabilities of deep learning with the use
of attentional mechanisms to support richly context-sensitive processing \citep{transformers,vaswani2017attention,
    kerg2020untangling}. However, it is clear that transformers are missing core capabilities required for modeling
human thought \citep{mahowald2023dissociating}.  In particular, they lack mechanisms required to emulate forms of
flexibility and efficiency exhibited by the human brain, including an ability to support analogy and abstraction.
While large language models show a surprising ability to complete some analogies \citep{webb}, this ability
emerges implicitly after processing vast amounts of data. The algorithmic challenge is to provide ways of binding
domain-specific information to low dimensional, abstract representations that can be used to compute a given function
in any setting for which it is relevant, based on limited data.

Our approach is motivated by a type of inductive bias for relational learning architectures we call the
``relational bottleneck," which is motivated by principles of cognitive neuroscience that shed light on the brain
subsystems involved when natural intelligence shows an ability to flexibly generalize abstract structure across
domains of processing. In particular, the framework of complementary learning systems \citep{McClelland:1995,
    Kumaran:2016} describes two distinct types of neural mechanisms for learning and memory around which the brain
is organized, implementing a tradeoff between slow, incremental forms of learning required to encode stable statistical
structure present in the environment (semantic memory), and the ability to rapidly encode and remember novel
associations (episodic memory).

Recently, it has been proposed that episodic memory may also serve to implement a relational bottleneck,
by allowing components of semantic memory that represent abstract functions to be separated from those
responsible for processing domain-specific information, while allowing the two types of components to be coupled
through rapid binding and similarity-based retrieval of representations \citep{esbn}.
This ``relational bottleneck'' imposes an inductive bias that constrains the flow of information from sensory or
motor subsystems to reasoning and decision making subsystems, by restricting this information to relations, as
computed through inner products between distributed representations.
%At the same time, it synergitically serves as a mechanism for variable binding, by rapidly associating domain-specific information (e.g., fillers) with abstract representations that can be used as arguments (i.e., roles)
%to general purpose functions appropriate for performing a task.
%removed for lack of space
In this paper we present a framework that recasts this inductive bias in terms of an extension of transformers, in
which specific types of attention mechanisms enforce the relational bottleneck. This creates a potentially powerful
combination of deep learning and relational learning that implements a form of symbolic processing, enabling
abstraction and generalization from limited data.

%Our approach was developed while thinking about the Emergent Symbol Binding Network (ESBN) framework \citep{esbn}, which can be seen as closely related to the neural Turing machine \citep{NTM}. Both frameworks augment traditional controllers such as LSTMs with external stores that can be viewed as computational models of episodic memory. However, the ESBN framework enforces a relational bottleneck by dividing the external memory into ``sensory'' and ``abstract'' sides, with lookups on the sensory side carried out using inner products. The key observation we develop in this paper is that by viewing episodic memory query-key lookups in terms of cross attention operations, the relational bottleneck can be naturally implemented in an extension of transformers. Our proposed \textit{Abstractor} framework extends ESBN to enable more complex, higher order forms of relational learning by combining it with both the attentional mechanisms and hierarchical structure of transformers. This creates a potentially powerful combination of deep learning and relational learning that enables abstraction and generalization from limited data.
