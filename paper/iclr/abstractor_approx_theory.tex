\section{Universal approximation of relation functions}\label{sec:abstractor_approx_theory}

In this section, we characterize the function class of the Abstractor and relational cross-attention. We will show that, in an appropriate sense, a single Abstractor layer can compute a sequence of abstract states $A = (A_1, \ldots, A_\m)$ such that $A_i$ approximates an arbitrary function of object $i$'s relations with the other objects in the input. Recall that relational cross-attention takes the form
\begin{equation*}
    A \gets \sigma_{\mathrm{rel}}\paren{\phi(X) {\psi(X)}^\top} S,
\end{equation*}
where $\phi, \psi: \calX \to \reals^{d_{\mathrm{proj}}}$ are multi-layer perceptrons, and $S$ is the matrix of symbols. In this analysis, we consider $\sigma_{\mathrm{rel}}: x \mapsto x$ to be the linear activation function. Further, we consider positional symbols. For simplicity, we will assume the single-head variant of relational cross-attention. The function class results derived here would of course carry over to the multi-head case, where each `head' can approximate a function in the function class.

Recall that an Abstractor module comprises several layers, each composed of relational cross-attention and a feedforward layer. Hence, the overall operation in one layer of an Abstractor module is
\begin{equation}\label{eq:abstractor_layer}
    \mathrm{AbstractorLayer}(X) = \mathrm{FeedForward}\paren{\phi(X) {\psi(X)}^\top \, S}.
\end{equation}
We will characterize the function class $\mathrm{AbstractorLayer}: \calX^\m \to \reals^{\m \times d}$ induced by varying the parameters of $\phi,\, \psi,\, \mathrm{FeedForward}$ and $S$.

\begin{remark}
    In~\Cref{eq:relational_crossattn}, we formulate relational cross-attention with the maps $\phi, \psi$ as linear projections, whereas $\phi, \psi$ are multi-layer perceptrons in~\Cref{eq:abstractor_layer}. However, recall that the input to each Abstractor layer is the output of the preceding layer which ends with a multi-layer perceptron. The function class of a multi-layer perceptron followed by two different linear projections is the same as the function class of two different multi-layer perceptrons. We focus on~\Cref{eq:abstractor_layer} for ease of presentation.
\end{remark}

The following result shows that a single-layer abstractor returns a sequence of abstract states which can approximate an arbitrary function of each object's relations with the other object in the input. The result is based on the analysis in~\citep{altabaaApproximationRelationFunctions2024} which characterizes the function class of inner products of neural networks.

\begin{theorem}
    Suppose $\calX$ is a compact Euclidean space. Let $r: \calX \times \calX \to \reals$ be any continuous relation function, and $f: \reals^\m \to \reals^d$ any continuous function. Consider the function $g: \calX^n \to \reals^{\m \times d}$ defined by
    \begin{equation*}
        (x_1, \ldots, x_\m) \mapsto \paren{f(R_1), \ldots, f(R_\m)},
    \end{equation*}
    where $R_i = {(r(x_i, x_j))}_{j \in [\m]}$ is the vector of $x_i$'s relations with the other objects in the input, according to $r$. Then, for any $\varepsilon > 0$, there exists MLPs $\phi, \psi, \mathrm{FeedForward}$ and a choice of symbols $S$ such that the Abstractor layer approximates $g$ in the sup-norm
    \begin{equation*}
        \infnorm{g(x_1, \ldots, x_\m) - \mathrm{AbstractorLayer}(x_1, \ldots, x_\m)} \leq \varepsilon, \quad \text{Lebesgue almost-everywhere.}
    \end{equation*}
\end{theorem}
\begin{proof}
    Let $(A_1, \ldots, A_\m) = \mathrm{AbstractorLayer}(x_1, \ldots, x_\m)$ be the abstract states returned by the abstractor layer. For each $i \in [\m]$,
    the abstract state $A_i$ takes the form,
    \begin{equation*}
        A_i = \mathrm{FeedForward}\paren{\sum_{j=1}^{\m} \iprod{\phi(x_i)}{\psi(x_j)} s_j}.
    \end{equation*}
    where $S = (s_1, \ldots, s_\m)$ are the symbols assigned to each object. Let the symbol dimension be $n$ and let $s_i = \bm{e}_i$, the canonical basis vectors. Then, $A_i = \mathrm{FeedForward}(R_i)$, where $R_i = {(\iprod{\phi(x_i)}{\psi(x_j)})}_{j \in [\m]} \in \reals^\m$. By~\citep[Theorem 3.1]{altabaaApproximationRelationFunctions2024}, there exists MLPs $\phi, \psi$ such that their inner product approximates any continuous relation function. In particular, for any $\varepsilon_1 > 0$, there exists $\phi, \psi$ such that
    \begin{equation}\label{eq:approx_proof_r_approx}
        \abs{r(x, y) - \iprod{\phi(x)}{\psi(y)}} \leq \varepsilon_1, \quad \text{Lebesgue almost-every } x,y \in \calX
    \end{equation}
    Let $\phi, \psi$ be such MLPs where $\varepsilon_1$ is to be determined later. Note that~\citep{altabaaApproximationRelationFunctions2024} gives a bound on the number of neurons needed in terms of the continuity of $r$ and the dimension of $\calX$. Similarly, by the universal approximation property of MLPs~\citep[e.g.,][]{cybenkoApproximationSuperpositions1989}, $f: \reals^{\m} \to \reals^{d}$ can be approximated by $\mathrm{FeedForward}$ uniformly in the sup-norm. That is, for any $\varepsilon_2$ there exists $\mathrm{FeedForward}$ such that
    \begin{equation}\label{eq:approx_proof_f_approx}
        \sup_{z \in \reals^\m} \infnorm{f(z) - \mathrm{FeedForward}(z)} \leq \varepsilon_2.
    \end{equation}

    Let $[R]_{ij} = r(x_i, x_j)$ and $[\hat{R}]_{ij} = \iprod{\phi(x_i)}{\psi(x_j)}$. Then, the difference $g(x_1, \ldots, x_\m) - \mathrm{AbstractorLayer}(x_1, \ldots, x_\m)$ is given by
    \begin{equation}\label{eq:approx_proof_diff}
        \Big[f(R_{1\cdot}), \ldots, f(R_{1\cdot})\Big] - \Big[\mathrm{FeedForward}(\hat{R}_{1\cdot}), \ldots, \mathrm{FeedForward}(\hat{R}_{\m\cdot})\Big]
    \end{equation}

    Note that $\hat{R}_{i\cdot}$ is close to $R_{i\cdot}$ by~\Cref{eq:approx_proof_r_approx}
    \begin{equation}\label{eq:approx_proof_R_approx}
        \begin{split}
            \infnorm{\hat{R}_{i\cdot} - R_{i\cdot}} &= \max_{j \in [\m]} \abs{\iprod{\phi(x_i)}{\psi(x_j)} - r(x_i, x_j)} \\
            &\leq \varepsilon_1 \quad \text{Lebesgue almost-everywhere}.
        \end{split}
    \end{equation}

    Now consider the $(i,j)$-th element of the difference in~\Cref{eq:approx_proof_diff}
    \begin{equation*}
        \abs{\mathrm{FeedForward}_i(\hat{R}_{j\cdot}) - f_i(R_{j\cdot})} \leq \abs{\mathrm{FeedForward}_i(\hat{R}_{j\cdot}) - f_i(\hat{R}_{j\cdot})} + \abs{f_i(\hat{R}_{j\cdot}) - f_i(R_{j\cdot})}
    \end{equation*}
    The first term is bounded by $\varepsilon_2$ due to~\Cref{eq:approx_proof_f_approx}. Let $\varepsilon_2 = \varepsilon / 2$. Recall that $f: \reals^n \to \reals^d$ is continuous, and hence for all $\epsilon > 0$ there exists $\delta_f(\varepsilon) > 0$ such that $\infnorm{z_1 - z_2} \leq \delta(\varepsilon) \implies \infnorm{f(z_1) - f(z_2)} \leq \varepsilon$. Letting $\varepsilon_1 = \delta(\varepsilon / 2)$, implies that the second term is bounded by $\varepsilon / 2$ Lebesgue almost-everywhere due to~\Cref{eq:approx_proof_R_approx}.

    This holds for all $i, j$, which completes the proof.

\end{proof}