\section{Introduction}

\awni{Question on title: "enhanced relational reasoning" vs "explicit relational reasoning" vs just "relational reasoning"? one reason to include a prefix is that transformers can model relations, but they are less sample-efficient due to their entangled representations}

\texttt{A paragraph on the importance of relations to cognition/intelligence.}

The ability to infer and process relations is at the heart of abstraction and creative thinking. Under one theory of cognitive science, intelligence can be understood through the lens of two categories of cognition---``neocortical'' and ``prefrontal''. Neocortical intelligence is refers to the ability to acquire semantic and procedural knowledge. By contrast, preforntal intelligence refers to the ability to identify novel associations and relations, building abstractions and generalizing to new domains. It is sometimes argued that modern deep learning systems achieve neocortical intelligence through efficient function approximation, but so far have limited prefrontal intelligence.

\texttt{A paragraph on how transformers fit into this theory. the attention mechanisms of transformers enables contextual processing. but they don't support explicit relational reasoning. The emerging literature on relational reasoning shows that the Transformer can be sample-inefficient in learning relational reasoning tasks.}

An example of this is the Transformer architecture \citep{vaswani2017attention}. The transformer is a powerful sequence model with the ability to . A testament to the power and versatility of the Transformer architecture is the fact that it is at the core of the most successful large language models today \citep{gpt3,gpt4,llamma,T5,etc?}. Beyond architectural variations, there is no real competitor to the Transformer in this domain. Recent work has studied the emergent abstraction abilities of large language models based on the Transformer architecture. While large language models show some ability to complete analogies~\citep{}, this is attained implicitly through processing vast amounts of data.

\texttt{A paragraph introducing our work and our main claims/contributions. relational bottleneck. Abstractor module.}

In this work we propose an extension of the Transformer framework for explicit relational reasoning through a novel module called the \textit{Abstractor}. The Abstractor achieves this enhanced relational reasoning through an inductive bias we call a \textit{relational information bottleneck}. This is an inductive bias which enforces that the representations learned by the Abstractor encode purely relational information which is abstracted away from the attributes of individual objects. At the core of the Abstractor module is a novel variant of attention called \textit{relational cross-attention}. This enables more focused and explicit relational reasoning, absent any extraneous sensory information. Additionally, it enables improved out-of-distribution generalization since the same relations may be present in different domains, even if the underlying sensory information of individual objects are different.

We empirically evaluate the Abstractor on two sets of tasks. The first set of tasks are based on learning order relations and sorting sequences of randomly-generated objects. This is a sequence-to-sequence relational task, which is so far unexplored in the literature on relational architectures (which has so far focused on discriminative tasks). We compare an Abstractor-based model to a standard transformer and observe dramatic improvements in sample efficiency. The second set of tasks are based on solving mathematical problems. Whereas the sorting tasks are purely relational synthetic tasks, the mathematical problem-solving tasks are more realistic and require a combination of relational reasoning and function approximation. Here too, the Abstractor yields a small but consistent improvement in sample efficiency over a standard transformer. This suggests that explicit relational reasoning is a useful architectural addition to sequence models. We hope that this work marks a step towards the development of machine learning methods and theories that for improved prefrontal intelligence.

\subsection{Related Work}

There exists an emerging literature on developing machine learning architectures with explicit relational reasoning capabilities. An early example is the Relation Network proposed in~\citep{santoro1}. The essential idea here is process pairwise relations by applying an MLP to the concatenation of object representations and aggregating the outputs by a simple summation. Given a sequence of objects $X = (x_1, \ldots, x_\m)$ as input, the Relation Network is given by $\mathrm{RN}(X) = f_\phi(\sum_{ij} g_\theta(x_i, x_j))$, where $f_\phi, g_\theta$ are MLPs.~\citep{shanahanExplicitlyRelationalNeural} proposes the PrediNet architecture which aims to learn propositional representations. The ESBN model proposed in~\citep{esbn} is an architecture which is similar in spirit to the present work. It separates sensory information from relational information and processes the relational information using an LSTM controller. Another architecture which is similar in spirit is the CoRelNet architecture proposed in~\citep{kerg2022neural}, which reduces relational learning to modeling a similarity matrix. It is given by, $\mathrm{MLP}(\mathrm{flatten}(R))$, where $R$ is the similarity matrix $R = \mathrm{Softmax}\paren{[\iprod{\phi(x_i)}{\phi(x_j)}]_{ij}}$ and $\phi$ is an encoder.

Prior work has made great strides in relational learning and has demonstrated that architectures with the right inductive biases can result in significant data-efficiency improvements for learning relational tasks. The Abstractor makes several important contributions to this line of work. First, the relational cross-attention mechanism of the Abstractor is a universal approximator for relation functions (see~\Cref{ssec:function_classes_preview}). In particular, it can model multi-dimensional and asymmetric relations. Moreover, existing work on relational architectures has focused on \textit{discriminative} relational tasks. The Abstractor framework naturally supports \textit{generative} sequence-to-sequence relational tasks. Finally, since the Abstractor framework is an extension of Transformers, it immediately inherits the strengths of Transformers in sequence-modeling. This enables the Abstractor to yield improvements in more general sequence-modeling tasks, such as natural language understanding, in addition to purely relational tasks.