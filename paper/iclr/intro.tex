\section{Introduction}

The ability to infer and process relations and reason in terms of analogies lies at the heart of human abilities for abstraction and creative thinking
\citep{snow,holyoak}. This capability is largely separate from our ability to acquire semantic and procedural knowledge through sensory tasks, such as image and audio processing. Modern deep learning systems can often capture this latter type of intelligence through efficient function approximation. However, deep learning has seen limited success with relational and abstract reasoning, which requires identifying novel associations from limited data and generalizing to new domains.

Recognizing the importance of this capability, machine learning research has explored several novel frameworks for relational learning \citep{TEM, NTM,episodicControl,shanahanExplicitlyRelationalNeural,esbn,mondal23learned,battaglia,barrett:2018,santoro1}. In this paper we propose a framework that casts relational learning in terms of Transformers. The success of Transformers lies in combining the function approximation capabilities of deep learning with the use of attentional mechanisms to support richly context-sensitive processing \citep{transformers,vaswani2017attention,kerg2020untangling}. However, it is clear that Transformers are missing core capabilities required for modeling human thought \citep{mahowald2023dissociating}, including an ability to support analogy and abstraction.While large language models show a surprising ability to complete some analogies \citep{webb}, this ability emerges implicitly after processing vast amounts of data.

The Transformer architecture has the ability to model relations between objects implicitly through attention mechanisms. However, we argue in this paper that standard attention produces entangled representations encoding a mixture of relational information and object-level information, resulting in suboptimal sample-efficiency for learning relations. The challenge is to provide ways of binding domain-specific information to low dimensional, abstract representations that can be used in a broader range of domains.

In this work we propose an extension of Transformers that enables explicit relational reasoning through a novel module called the \textit{Abstractor}. At the core of the Abstractor module is a variant of attention called \textit{relational cross-attention}. Our approach is motivated by an architectural inductive bias for relational learning we call the ``relational bottleneck," which separates relational information from extraneous object-level information, thus enabling more focused and explicit relational reasoning. Through the relational cross-attention mechanism, the Abstractor architecture creates a powerful combination of deep learning and relational learning enabling abstraction and generalization from limited data.

Our empirical evaluation is split into three sections. In the first, we evaluate the Abstractor on simple discriminative relational tasks and compare to existing relational architectures (which so far have focused on discriminative relational tasks). In the second, we evaluate the Abstractor on a purely relational sequence-to-sequence task---sorting sequences of randomly generated objects. These experiments give us a controlled setting in which to evaluate the Abstractor's ability to model relations. We observe that the Abstractor achieves a dramatic improvement in sample efficiency compared to a standard Transformer. In the third section, we evaluate the Abstractor on a more realistic task which requires a combination of relational reasoning as well as more general sequence modeling---solving mathematical problems. We observe that the Abstractor yields modest but consistent improvements in performance and sample efficiency over a standard Transformer. This provides evidence that the Abstractor module for relational reasoning is a useful architectural addition to sequence models.

\subsection{Related Work}

There exists an emerging literature on developing machine learning architectures with explicit relational reasoning capabilities. An early example is the Relation Network proposed in~\citep{santoro1}. The essential idea here is process pairwise relations by applying an MLP to the concatenation of object representations and aggregating the outputs by a simple summation. Given a sequence of objects $X = (x_1, \ldots, x_\m)$ as input, the Relation Network is given by $\mathrm{RN}(X) = f_\phi(\sum_{ij} g_\theta(x_i, x_j))$, where $f_\phi, g_\theta$ are MLPs.~\citep{shanahanExplicitlyRelationalNeural} proposes the PrediNet architecture which aims to learn propositional representations. The ESBN model proposed in~\citep{esbn} is an architecture which is similar in spirit to the present work. It separates sensory information from relational information and processes the relational information using an LSTM controller. Another architecture which is similar in spirit is the CoRelNet architecture proposed in~\citep{kerg2022neural}, which reduces relational learning to modeling a similarity matrix. It is given by, $\mathrm{MLP}(\mathrm{flatten}(R))$, where $R$ is the similarity matrix $R = \mathrm{Softmax}\paren{[\iprod{\phi(x_i)}{\phi(x_j)}]_{ij}}$ and $\phi$ is an encoder.

Prior work has made great strides in relational learning and has demonstrated that architectures with the right inductive biases can result in significant data-efficiency improvements for learning relational tasks. The Abstractor makes several important contributions to this line of work. First, the relational cross-attention mechanism of the Abstractor is a universal approximator for relation functions (see~\Cref{ssec:function_classes_preview}). In particular, it can model multi-dimensional and asymmetric relations. Moreover, existing work on relational architectures has focused on \textit{discriminative} relational tasks. The Abstractor framework naturally supports \textit{generative} sequence-to-sequence relational tasks. Finally, since the Abstractor framework is an extension of Transformers, it immediately inherits the strengths of Transformers in sequence-modeling. This enables the Abstractor to yield improvements in more general sequence-modeling tasks, such as natural language understanding, in addition to purely relational tasks.