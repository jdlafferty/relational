\documentclass[12pt,pdftex,noinfoline]{imsart}
\RequirePackage[OT1]{fontenc}
\usepackage{amsthm,amsbsy,amsmath,amsfonts,natbib,mathtools,amssymb,mathrsfs}
\RequirePackage{hypernat}
\usepackage{times}
\usepackage{hyperref}
\usepackage[usenames,dvipsnames,svgnames,table]{xcolor}
\hypersetup{citecolor=MidnightBlue}
\hypersetup{linkcolor=MidnightBlue}
\hypersetup{urlcolor=MidnightBlue}
\usepackage{enumerate}
\usepackage{fullpage}
\usepackage[margin=1in,footskip=.40in]{geometry}
\usepackage{dsfont}
\usepackage{wrapfig}
% \usepackage{url}            % simple URL typesetting
\usepackage{xurl} % this allows line-breaking in urls
\usepackage{booktabs}       % professional-quality tables
\usepackage{caption}
\usepackage{subcaption}
\usepackage{graphicx}
\usepackage{verbatim}

\usepackage{parskip}

\usepackage{mymathstyle} % math style with a bunch of commands defined

\usepackage{cleveref}
\usepackage{algorithm2e}
\usepackage{graphicx}
\crefname{algocf}{alg.}{algs.}
\Crefname{algocf}{Algorithm}{Algorithms}
\RestyleAlgo{ruled}

% \numberwithin{equation}{section}
% \newtheorem{thm}{Theorem}[section]
% \newtheorem{result}{Result}
% \newtheorem{lemma}{Lemma}[section]
% \newtheorem{corollary}{Corollary}[section]
% \newtheorem{prop}{Proposition}[section]
% \theoremstyle{remark}
% \newtheorem{example}{Example}[section]
% \newtheorem{remark}{Remark}[section]
\newtheorem{result}{Result}
\newcommand{\m}{m} % define a macro for the number of objects

%region
% john's definitions
\newtheorem*{thm1}{Claim A}
\newtheorem*{thm2}{Claim B}
\newtheorem*{thm3}{Claim C}
\newtheorem*{thm4}{Claim D}
\def\conditionA{Condition $A$}
\def\conditionAp{Condition $\tilde A$}
\def\conditionB{Condition $B$}
\newtheorem*{con0}{\conditionA}
\newtheorem*{con1}{\conditionAp}
\newtheorem*{con2}{\conditionB}

\newcommand{\MLP}{\text{\sc mlp}}
\newcommand{\FeedForward}{\text{FeedForward}}
\newcommand{\Softmax}{\text{Softmax}}
% \newcommand{\argmin}{\mathop{\rm argmin}}
\newcommand{\ess}{\mathop{\rm ess}}
% \newcommand{\argmax}{\mathop{\rm argmax}}
\newcommand{\lnorm}[2]{\|{#1} \|_{{#2}}}
\newcommand{\indc}[1]{{\mathbf{1}_{\left\{{#1}\right\}}}}
%\newcommand{\norm}[1]{\|{#1} \|}
% \newcommand{\norm}[1]{\left\|{#1} \right\|}
\newcommand{\hf}{{1/2}}
\newcommand{\wh}{\widehat}
\newcommand{\wt}{\widetilde}
\newcommand{\Norm}[1]{\|{#1} \|}
\newcommand{\Fnorm}[1]{\lnorm{#1}{\rm F}}
% \newcommand{\fnorm}[1]{\|#1\|_{\rm F}}
% \newcommand{\opnorm}[1]{\|#1\|_{\rm op}}
\newcommand{\nnorm}[1]{\|#1\|_{\rm *}}
% \newcommand{\Prob}{\mathbb{P}}
% \newcommand{\Expect}{\mathbb{E}}
% \newcommand{\Span}{{\rm span}}
\newcommand{\rank}{\mathop{\sf rank}}
\newcommand{\sgn}{\mathop{\rm sign}}
\newcommand{\rt}{\mathop{\sf root}}
\newcommand{\R}{\mathop{\sf R}}
% \newcommand{\Tr}{\mathop{\sf Tr}}
\newcommand{\diag}{\mathop{\text{diag}}}
\newcommand{\supp}{{\rm supp}}
% \newcommand{\iprod}[2]{\left \langle #1, #2 \right\rangle}
\newcommand{\pth}[1]{\left( #1 \right)}
\newcommand{\qth}[1]{\left[ #1 \right]}
\newcommand{\sth}[1]{\left\{ #1 \right\}}
% \newcommand{\calL}{\mathcal{L}}
\newcommand{\floor}[1]{{\left\lfloor {#1} \right \rfloor}}
% \newcommand{\TV}{{\sf TV}}
\newcommand{\blue}{\color{blue}}
\newcommand{\nb}[1]{\texttt{\blue[#1]}}

\def\bra#1{\langle #1 \vert}
\def\ket#1{\vert #1 \rangle}
\def\braket#1#2{\langle #1 \vert #2 \rangle}
%\def\bind#1#2{\ket{#1}\bra{#2}}
\def\bind#1#2{\ket{#1}|\ket{#2}}
\def\bind#1#2{#1 \| #2}
\let\tilde\widetilde
\def\softmax#1{\mbox{Softmax}\left(#1\right)}
\def\task#1{\vskip5pt\noindent{\it\bfseries #1.}}
\def\qkv#1#2#3{\mbox{CrossAttention}(Q\!\leftarrow\!#1,\; K\!\leftarrow\!#2,\; V\!\leftarrow\!#3)}
\def\selfattention#1{\mbox{SelfAttention}(#1)}
\def\crossattention#1#2#3{\qkv{#1}{#2}{#3}}
\def\crossattend#1#2#3#4{\mbox{CrossAttention}_{#4}(Q\!\leftarrow\!#1,\; K\!\leftarrow\!#2,\; V\!\leftarrow\!#3)}
\makeatletter
\newcommand*\bigcdot{\mathpalette\bigcdot@{.5}}
\newcommand*\bigcdot@[2]{\mathbin{\vcenter{\hbox{\scalebox{#2}{$\m@th#1\bullet$}}}}}
\makeatother

%endregion

% commands for john and i to leave comments
\newcommand{\awni}[1]{\textcolor{blue}{\texttt{[Awni]}: #1}}
\newcommand{\john}[1]{\textcolor{red}{\texttt{[John]}: #1}}

% \title{Abstractors and relational cross-attention: An inductive bias for explicit relational reasoning in Transformers}

\begin{document}

\def\snote#1{${}^{#1}$}
\setlength{\parskip}{0.5em}
\begin{frontmatter}
{\bf\Large Abstractors and relational cross-attention: 
An inductive bias \\[8pt] for explicit relational reasoning in Transformers}
%\affil[**]{Department of Statistics and Data Science, Yale University}
\begin{aug}
\vskip15pt
\address{
\begin{tabular}{ccccc}
{\normalsize\rm\bfseries Awni Altabaa}\snote{1} & {\normalsize\rm\bfseries Taylor Webb}\snote{2} & {\normalsize\rm\bfseries Jonathan  Cohen}\snote{3} & {\normalsize\rm\bfseries John Lafferty}\snote{4}\\[5pt]
\end{tabular}
\vskip5pt
\footnotetext{
\snote{1}Department of Statistics and Data Science, Yale University; awni.altabaa@yale.edu.
\snote{2}Department of Psychology, UCLA; taylor.w.webb@gmail.com.
\snote{3}Department of Psychology and Princeton Neuroscience Institute, Princeton University; jdc@princeton.edu.
\snote{4}Department of Statistics and Data Science, Wu Tsai Institute, Institute for Foundations of Data Science, Yale University; john.lafferty@yale.edu.
}
\today
\vskip10pt
}
\begin{abstract}
    An extension of Transformers is proposed that enables explicit relational reasoning through a novel module called the \textit{Abstractor}.
    At the core of the Abstractor module is a variant of attention called \textit{relational cross-attention}.
    The approach is motivated by an architectural inductive bias for relational learning called the ``relational bottleneck," 
    which separates relational information from extraneous sensory information, to  enable more focused and explicit relational reasoning,
    leading to abstraction and generalization from limited data.
    In addition to evaluating the Abstractor on simple discriminative relational tasks and comparing to existing relational architectures, 
    we also evaluate the Abstractor on relational sequence-to-sequence tasks, 
    observing dramatic improvements in sample efficiency compared to a standard Transformer. We also evaluate the Abstractor on a collection of mathematics problems that benefit from a combination of relational reasoning and more traditional sequence modeling.
 \end{abstract}
\end{aug}
\end{frontmatter}


\begin{abstract}
    An extension of Transformers is proposed that enables explicit relational reasoning through a novel module called the \textit{Abstractor}.
    At the core of the Abstractor module is a variant of attention called \textit{relational cross-attention}.
    The approach is motivated by an architectural inductive bias for relational learning called the ``relational bottleneck," 
    which separates relational information from extraneous sensory information, to  enable more focused and explicit relational reasoning,
    leading to abstraction and generalization from limited data.
    In addition to evaluating the Abstractor on simple discriminative relational tasks and comparing to existing relational architectures, 
    we also evaluate the Abstractor on relational sequence-to-sequence tasks, 
    observing dramatic improvements in sample efficiency compared to a standard Transformer. We also evaluate the Abstractor on a collection of mathematics problems that benefit from a combination of relational reasoning and more traditional sequence modeling.
 \end{abstract}
\section{Introduction}

The ability to infer and process relations and reason in terms 
of analogies lies at the heart of human abilities for abstraction and creative thinking
\citep{snow,holyoak}. This capability is 
largely separate from our ability to acquire semantic and procedural 
knowledge through sensory tasks, such as image and audio processing. Modern 
deep learning systems can often capture this latter type of intelligence 
through efficient function approximation. However, deep learning has 
seen limited success with relational and abstract reasoning, which 
requires identifying novel associations from limited data
and generalizing to new domains. 

Recognizing the importance of this capability,
machine learning research has explored several novel frameworks for relational learning 
\citep{TEM, NTM,episodicControl,shanahanExplicitlyRelationalNeural,esbn,mondal23learned,battaglia,barrett:2018,santoro1}. 
In this paper we propose a framework that casts relational learning in terms of Transformers. 
The success of Transformers lies in combining the function approximation capabilities of deep learning with the use
of attentional mechanisms to support richly context-sensitive processing \citep{transformers,vaswani2017attention,
    kerg2020untangling}. However, it is clear that Transformers are missing core capabilities required for modeling
human thought \citep{mahowald2023dissociating}, including an ability to support analogy and abstraction.
While large language models show a surprising ability to complete some analogies \citep{webb}, this ability
emerges implicitly after processing vast amounts of data.

The Transformer architecture has the ability to model relations between objects implicitly through attention mechanisms. However, we argue in this paper that standard attention produces entangled representations encoding a mixture of relational information and sensory information, resulting in suboptimal sample-efficiency for learning relations. The challenge is to provide ways of binding
domain-specific information to low dimensional, abstract representations 
% that can be used to compute a given function in any setting for which it is relevant, based on limited data.
that can be used in a broader range of sensory domains.

In this work we propose an extension of Transformers that enables explicit relational reasoning through a novel module called the \textit{Abstractor}.
At the core of the Abstractor module is a novel variant of attention called \textit{relational cross-attention}.
% This enables more focused and explicit relational reasoning, separating relational information from extraneous sensory information.
Our approach is motivated by an architectural inductive bias for relational learning we call the
``relational bottleneck," 
which separates relational information from extraneous sensory information, thus enabling more focused and explicit relational reasoning.
% which restricts the flow of information from sensory subsystems to  
% reasoning subsystems to be relational. % To-Do: edit this sentence?
Through the relational cross-attention mechanism, the Abstractor architecture
creates a powerful combination of deep learning and relational learning
% that implements a form of symbolic processing,
enabling abstraction and generalization from limited data.
% Additionally, it enables improved out-of-distribution generalization since the same relations may be present in different domains, even if the underlying sensory information of individual objects is different.

Our empirical evaluation is split into three sections. In the first, we evaluate the Abstractor on simple discriminative relational tasks and compare to existing relational architectures (which so far have focused on discriminative relational tasks).
In the second, we evaluate the Abstractor on a purely relational sequence-to-sequence task---sorting sequences of randomly generated objects. These experiments give us a controlled setting in which to evaluate the Abstractor's ability to model relations. We observe that the Abstractor achieves a dramatic improvement in sample efficiency compared to a standard Transformer.
In the third section, we evaluate the Abstractor on a more realistic task which requires a combination of relational reasoning as well as more general sequence modeling---solving mathematical problems. We observe that the Abstractor yields modest but consistent improvements in performance and sample efficiency over a standard Transformer. This provides evidence that the Abstractor module for relational reasoning is a useful architectural addition to sequence models.

% We empirically evaluate the Abstractor on two sets of tasks. The first set of tasks is based on learning order relations and sorting sequences of objects. This is a sequence-to-sequence relational task, which is so far unexplored in the literature on relational architectures, which had 
% previously focused on discriminative tasks. We compare an Abstractor-based model to a standard Transformer and observe dramatic improvements in sample efficiency. The second set of tasks is based on solving mathematical problems. Whereas the sorting tasks are purely relational synthetic tasks, the mathematical problem-solving tasks are more realistic and require a combination of relational reasoning and function approximation. Here too, the Abstractor yields modest but consistent improvements in performance and sample efficiency over a standard Transformer. This provides
% evidence that the Abstractor module for relational reasoning is a useful architectural addition to sequence models. 

\subsection{Related Work}\label{ssec:related_work}
A growing body of literature is focused on developing machine learning architectures with explicit relational reasoning capabilities. An early example is the Relation Network proposed in~\citep{santoro1}. The essential idea here is process pairwise relations by applying an MLP to the concatenation of object representations and aggregating the outputs by a simple summation. Given a sequence of objects $X = (x_1, \ldots, x_\m)$ as input, the Relation Network is given by $\mathrm{RN}(X) = f_\phi(\sum_{ij} g_\theta(x_i, x_j))$, where $f_\phi, g_\theta$ are MLPs.~\citep{shanahanExplicitlyRelationalNeural} proposes the PrediNet architecture which aims to learn representations of relations which are inspired by predicate logic. The ESBN model proposed in~\citep{esbn} is a memory-augmented LSTM network, inspired by ideas from cognitive science, which aims to factor representations into `sensory' and `relational'. In this sense, it is similar in spirit to the present work. Another architecture which is similar in spirit is the CoRelNet architecture proposed in~\citep{kerg2022neural}, which reduces relational learning to modeling a similarity matrix. It is given by, $\mathrm{MLP}(\mathrm{flatten}(R))$, where $R$ is the similarity matrix $R = \mathrm{Softmax}\paren{[\iprod{\phi(x_i)}{\phi(x_j)}]_{ij}}$ and $\phi$ is an encoder.

The Transformer~\citep{vaswani2017attention} is a common baseline which is compared against in this literature~. It is shown in these works that explicitly relational architectures outperform the Transformer, sometimes by large margins, on several synthetic discriminative relational tasks~\citep{shanahanExplicitlyRelationalNeural,esbn,kerg2022neural}. In this work, we offer an explanation, arguing that while the Transformer architecture is versatile enough to learn such relational tasks given enough data, it does not support relational reasoning explicitly. The Abstractor module extends the Transformer framework by learning representations of relations which are disentangled from extraneous features about individual objects. Our experiments first validate that the Abstractor, on its own, achieves the same sample-efficiency gains of other relational architectures on discriminative relational task. We then evaluate whether the Abstractor can augment a Transformer to improve relational reasoning by evaluating on synthetic \textit{sequence-to-sequence} relational tasks, which has so far been unexplored in the literature on explicitly relational architectures. Finally, we evaluate an Abstractor-based architecture on a more realistic mathematical problem-solving task to evaluate the potential of the idea on more general tasks.
\section{Relational cross-attention and the abstractor module}\label{sec:abstractor_module}

At a high level, the primary function of an Abstractor is to compute abstract relational features of its inputs.\footnote{In this paper, we will tend to use the name `Abstractor' to refer to both the module and to model architectures which contain the Abstractor module as a main component.} That is, given a set or sequence of input objects $x_1, \ldots, x_\m$, the relational abstractor learns to model a relation $r(\cdot, \cdot)$ and computes a function on the set of pairwise relations between objects ${\{ r(x_i, x_j) \}}_{ij}$. At the heart of the abstractor module is an inductive bias we call the \textit{relational bottleneck}---it separates relational information from the features of individual objects.

%The relations modeled by the Abstractor and the computations on them are learned to carry out a specific prediction task. This learning is often end-to-end, but, crucially, the Abstractor framework naturally supports modular learning.

\subsection{Modeling relations as inner products}\label{ssec:relations_as_inner_prods}

A relation function is a function which maps a pair of objects $x_1, x_2 \in \calX$ to a vector representing the relation between the two objects. We model pairwise relations as inner products between appropriately encoded (or `filtered') object representations. In particular, we model the pairwise relation function $r(\cdot, \cdot) \in \mathbb{R}^{d_r}$ in terms $d_r$ learnable `left encoders' $\phi_1, \ldots, \phi_{d_r}$, and $d_r$ `right encoders' $\psi_1, \ldots, \psi_{d_r}$,
\begin{equation}\label{eq:inner_prod_rel}
    r(x_1,x_2) = \left(\langle \phi_1(x_1), \psi_1(x_2) \rangle, \langle \phi_2(x_1), \psi_2(x_2) \rangle, \ldots, \langle \phi_{d_r}(x_1), \psi_{d_r}(x_2) \rangle \right)^\top \in \mathbb{R}^{d_r}.
\end{equation}
A common choice is to model $(\phi_i, \psi_i)_{i\in [d_r]}$ as linear or affine maps, perhaps with a common non-linear encoder as a first step. Considering all pairwise relations yields a \textit{relation tensor}, $R = \left[r(x_i, x_j)\right]_{i,j} \in \mathbb{R}^{\m \times \m \times d_r}$.

% NOTE [awni]: could remove reference to Santoro if we're tight on space?
In principle, relations could be modeled by an arbitrary learnable function applied to the concatenation of the features of a pair of objects.~\citep{santoro1} takes this approach and models relations through MLPs applied to the concatenation of the features of a pair of objects. This approach is versatile in principle and can work given enough data. However, modeling relations as inner products has some important advantages. First, it induces a pressure on the resultant representations to encode relational information and constricts the leakage of information about individual objects. When relations are modeled as $g_\theta(\mathrm{concat}(x_1, x_2))$, for some parameterized function class $g_\theta$, there is no pressure or restriction that this function represents relations between the two objects---it can just as well represent information about the objects' features individually.

Modeling relations as inner products $\iprod{\phi(x_1)}{\psi(x_2)}$ ensures that the output represents a comparison between the two objects' features. More precisely, inner product relations induce a geometry on the object space $\calX$. To see this, consider the case of symmetric relations. Then, the inner product $\iprod{\phi(x_1)}{\phi(x_2)}$ induces well-defined notions of distance, angles, and orthogonality in $\calX$. Finally, modeling relations as inner products does not result in a loss of expressive power since inner product relations capture arbitrary continuous functions on $\calX \times \calX$ (see~\Cref{ssec:universal_approximation}). Modeling relations as inner products is an approach that has been explored in previous work on relational architectures (e.g.,~\citep{esbn,kerg2022neural}) and has been shown to be a useful inductive bias on several discriminative relational tasks.

\subsection{Relational Cross-Attention}\label{ssec:relational_crossattention}

The core operation in a transformer is attention. For an input sequence $X = \paren{x_1, \ldots, x_\m}$, self-attention transforms the sequence via,
\begin{equation}\label{eq:self_attn}
    X' \gets \phi_v(X) \, \mathrm{Softmax}\paren{\phi_q(X)^\top \phi_k(X)},
\end{equation}
where $\phi_q, \phi_k, \phi_v$ are functions on $\calX$ applied elementwise to each object in the sequence (i.e., $\phi(X) = \paren{\phi(x_1), \ldots, \phi(x_\m)}$). Typically, those are linear or affine functions, with $\phi_q, \phi_k$ having the same dimensionality so we can take their inner product. Note that $\phi_q(X)^\top \phi_k(X)$ is a relation matrix in the sense defined above. Self-attention admits an interpretation as a form of message-passing. In particular, let $R = \mathrm{Softmax}\paren{\phi_q(X)^\top \phi_k(X)}$ be the softmax-normalized relation matrix. Then self-attention takes the form
\begin{equation}\label{eq:self_attn_message_passing}
    x_i' \gets \mathrm{MessagePassing}\paren{\set{(\phi_v(x_j), R_{ij})}_{j \in [\m]}} = \sum_{j} R_{ij} \phi_v(x_j).
\end{equation}

Thus, self-attention is a form of message-passing where the message from object $j$ to object $i$ is an encoding of object $j$'s features weighted by the (softmax-normalized) relation between the two objects. As a result, the processed representation obtained by self-attention involves some relational information. However, this relational information is entangled with the features of individual objects.

Our goal is to learn relational representations which are abstracted away from the features of individual objects in order to achieve more sample-efficient learning and improved generalization in relational reasoning. This is not naturally supported by standard self-attention, as it produces representations which entangle relational information with object-level attributes. We achieve this via a simple modification of attention---we replace the values $\phi_v(x_i)$ with input-independent vectors which identify the objects but do not encode any information about their features. We call those vectors \textit{symbols}. Hence, we make it so that the message from object $j$ to object $i$ is the symbol identifying object $j$, $s_j$, together with the relation between the two objects, $R_{ij}$, through
\begin{equation}\label{eq:relational_crossattn_message_passing}
    A_i \gets \mathrm{MessagePassing}\paren{\set{(s_j, R_{ij})}_{j \in [\m]}} = \sum_{j} R_{ij} s_j.
\end{equation}
Symbols act as abstract references to objects. They do not contain any information about the contents or features of the objects, but rather identify objects via their \textit{position}. The symbols $S = (s_1, s_2, \ldots)$ can be either learned parameters of the model or nonparametric positional embeddings (e.g., sinusoidal positional embeddings). Those vectors are `symbols' in the same sense that $x$ is a symbol in an equation like $y = x^2$---they are a reference to an unspecified value. The difference is that the symbols in relational cross-attention are distributed representations (i.e., vectors), so this gives a novel perspective on the long-standing problem of neural versus symbol computation.

This modification yields a variant of attention which we call \textit{relational cross-attention},
\begin{equation}\label{eq:relational_crossattn}
    X' \gets S_{1:m} \, \sigma_{\mathrm{rel}}\paren{\phi(X)^\top \psi(X)},
\end{equation}
where $S$ are the input-independent symbols, $\sigma_{\mathrm{rel}}$ is the relation activation function, and $\phi, \psi$ correspond to the query and key transformations. When the relation activation function $\sigma_{\mathrm{rel}}$ is softmax, this corresponds to $\mathrm{Attention}(Q \gets X,\, K \gets X,\, V \gets S)$. By contrast, self-attention corresponds to $\mathrm{Attention}(Q \gets X,\, K \gets X,\, V \gets X)$, mixing relational information with attributes of individual objects.

We observe in our experiments that allowing $\sigma_{\mathrm{rel}}$ to be a configurable hyperparameter can lead to performance benefits in some tasks. Softmax has the effect of normalizing the relation between a pair of objects $(i,j)$ based on the strength of $i$'s relations with the other objects in the sequence. In some tasks this is useful. In other tasks, this may mask relevant information. In that case, tanh, sigmoid, or no activation may be more appropriate.

Relational cross-attention implements a type of information bottleneck, which we term the relational bottleneck, wherein the resultant representation encodes only relational information about the object sequence and does not encode information about the features of individual objects\footnote{The diagonal entries of the relation matrix $R_{ii} = \iprod{\phi(x_i)}{\psi(x_i)}$ encode information about individual objects (a self-relation). This can be interpreted as a leakage of non-relational information. It's possible to implement a stricter relational bottleneck by masking the diagonal entries of $R$, but we find this to be unnecessary in our experiments.}. This enables a branch of the model to focus purely on modeling the relations between objects, yielding greater sample-efficiency in tasks which rely on relational reasoning.

% \begin{remark}
%     We call this operation ``relational cross-attention'' because 1) it computes relational representations, and 2) when attending to the input objects,  only
%     relational information, not sensory information, is allowed to cross over, since 
%     the values are input-independent. \awni{I was thinking (1) explains `relational' and (2) explains `cross-attention'. But we might want to remove this since we're tight on space and it's not really necessary.}
% \end{remark}

In our experiments, $\phi_i, \psi_i$ are linear maps $W_1^{(i)}, W_2^{(i)}$. Relational cross-attention is given by
\begin{equation}
    \begin{split}
        \mathrm{RelationalCrossAttention}(X, S) &= \mathrm{MultiHeadAttention}\paren{Q \gets X,\, K \gets X,\, V \gets S} \\
        &= W_o \, \mathrm{concat}\paren{A^{(1)}, \ldots, A^{(d_r)}}, \\
        \text{where } A^{(i)} &= W_o^{(i)} S \, \sigma_{\mathrm{rel}}\paren{(W_1^{(i)} X)^\top (W_2^{(i)} X)}.
    \end{split}
\end{equation}

\begin{figure}
    \begin{subfigure}[b]{0.5\textwidth}
        \centering
        \includegraphics[width=\textwidth]{figures/self_attn_fig.pdf}
        \caption{$E \gets \mathrm{SelfAttention}(X)$}
        \label{fig:self_attention}
    \end{subfigure}
    \hfill
    \begin{subfigure}[b]{0.5\textwidth}
        \centering
        \includegraphics[width=\textwidth]{figures/rel_crossattn_fig.pdf}
        \caption{$A \gets \mathrm{RelationalCrossAttention(X, S)}$}
        \label{fig:relational_cross_attention}
    \end{subfigure}
    \caption{Comparison of relational cross-attention with self-attention. Red represents object-level features, blue represents relational features, and purple represents entangled representations of both. Relational cross-attention computes relational information disentangled from the features of individual objects.}\label{fig:attn_mechanisms}
\end{figure}

\subsection{Abstractor module}\label{ssec:abstractor_module}

We now describe the Abstractor module. Like the Encoder in a Transformer, this is a module which processes an input sequence of objects $X = \paren{x_1, \ldots, x_\m}$ producing a a processed sequence of objects $A = \paren{A_1, \ldots, A_\m}$ which represents features of the input sequence. In the Encoder, the output objects $E = \paren{E_1, \ldots, E_\m}$ represents a mix of object-level attributes and relational information. In the Abstractor, $A$ represents purely relational information which is abstracted away from the features of individual objects. The core operation in an Abstractor module is relational cross-attention. Mirroring an encoder, an Abstractor module can consist of several layers, each consisting of relational cross-attention followed by a feedforward network. Optionally, we might apply a residual connection and layer-normalization as suggested in~\citep{vaswani2017attention}. The algorithmic description is presented in~\Cref{alg:abstractor_module}.

\begin{algorithm}[ht!]
	\caption{Abstractor module}\label{alg:abstractor_module}
	\SetKwInOut{Input}{Input}
	% \SetKwInOut{Output}{Output}
	% \SetKwInOut{LearnableParams}{Learnable parameters}
	% \SetKwInOut{HyperParams}{Hyperparameters}

	\Input{object sequence: $\bm{X} = (x_1, \ldots, x_\m) \in \reals^{d \times \m}$ }
	% \HyperParams{Dimension of relation $d_r$, Projection dimension $d_p$, dimension of embedding $d_e$}
	% \LearnableParams{projection matrices $W_1^{(i)}, W_2^{(i)} \in \reals^{d_p \times d_e}, \ i=1, \ldots, d_r$, parameters of feedforward networks}
	\vspace{1em}

    $A^{(0)} \gets S_{1:\m}$

	\For{\(l \gets 1\) \KwTo \(L\)}{

        $A^{(l)} \gets \mathrm{RelationalCrossAttention}\paren{X, A^{(l-1)}}$

        $A^{(l)} \gets A^{(l)} + A^{(l-1)}$ \quad \texttt{residual connection (optional)}

        $A^{(l)} \gets \mathrm{LayerNorm}(A^{(l)})$ \quad \texttt{(optional)}

        $A^{(l)} \gets \mathrm{FeedForward}\paren{A^{(l)}}$

        % \vspace{1em}
        % \texttt{optionally, self-attention:}

        % $A^{(l)} \gets \mathrm{SelfAttention}\paren{A^{(l)}}$

        % $A^{(l)} \gets A^{(l)} + A^{(l-1)}$ \quad \texttt{residual connection (optional)}

        % $A^{(l)} \gets \mathrm{LayerNorm}(A^{(l)})$ \quad \texttt{(optional)}

        % $A^{(l)} \gets \mathrm{FeedForward}\paren{A^{(l)}}$
    }

    \textbf{Output:} $A^{(L)}$

\end{algorithm}

The hyperparamters of an Abstractor module includes the the number of layers $L$, the relation dimension $d_r$ (i.e., number of heads),  the projection dimension $d_p$ (i.e., key dimension), the relation activation function $\sigma_{\mathrm{rel}}$, the dimension of the symbols $d_s$, the parameters of the feedforward network, whether to apply a residual connection, and whether to apply layer-normalization. The learnable parameters, at each layer, are the projection matrices $W_1^{(i)}, W_2^{(i)} \in \reals^{d_p \times d_s}, \ i \in [d_r]$, the symbols $S = \paren{s_1, s_2, \ldots } \in \reals^{d_s \times \texttt{max\_len}}$, and the parameters of the feedforward network. In our experiments, we use a 2-layer feedforward network with a hidden dimension $d_{\mathrm{ff}}$ and ReLU activation. The implementation in the publicly available code includes a few additional hyperparameters including whether to restrict the learned relations to be symmetric (via $W_1^{(i)} = W_2^{(i)}$), and whether to additionally apply self-attention after relational cross-attention. The symbols in the Abstractor can be either learned parameters or nonparametric positional embeddings (e.g., sinusoidal positional embeddings).

Observe that in the first layer of the Abstractor, the values in relational cross-attention are the input-independent symbols. In the following layers, the values are the abstract states from the previous layer. In the next subsection we characterize the function class of 1-layer Abstractor modules. Computations in deeper Abstractor modules are more difficult to interpret in terms of the original input sequence, but, empirically, we find that deeper Abstractor modules can yield improvements in performance.

\begin{remark}[Length generalization]\rm
    Similar to positional embeddings in standard Transformers, the symbols $S = (s_1, s_2, \ldots)$ can be either learned parameters of the model or non-parameteric positional embeddings (e.g., sinusoidal positional embeddings). In principle, using non-parameteric positional embeddings as symbols allows for generalization to longer sequence lengths than was observed during training. However, length-generalization remains an unsolved challenge in Transformers~\citep{kazemnejadImpactPositionalEncoding2023}. Although we don't carry-out a systematic evaluation of length-generalization for Abstractor-based models, it is likely that the same challenges apply. To begin to address this, we propose a variant of the Abstractor which uses position-relative symbols, $S = \paren{\ldots, s_{-1}, s_0, s_1, \ldots}$, where the message-passing operation of relational cross-attention becomes
    \begin{equation}\label{eq:position_relative_symbols}
            A_i \gets \mathrm{MessagePassing}\paren{\set{(s_{j-i}, R_{ij})}_{j \in [\m]}}
            = \sum_{j} R_{ij} s_{j-i}.
    \end{equation}
    Hence, the symbols carry information about relative-position with respect to the object being processed, as opposed to absolute position. In Transformers, relative positional embeddings have been shown to yield improvements in length-generalization.
\end{remark}

%\subsection{Universal approximation of relation functions for Abstractors}\label{ssec:universal_approximation}

In this section, we consider the representational power of the Abstractor module for computing functions of relations between a sequence of objects. We defer formal statements and proofs to the appendix. We show that the Abstractor module can approximate arbitrary functions of relations between objects, and that the Abstractor module can be composed to compute higher-order relations. Consider a 1-layer single-head Abstractor acting on a sequence of objects $X = \paren{x_1, \ldots, x_\m} \in \calX^\m$. Let $\sigma_\mathrm{rel}$ be the identity. Then, a 1-layer Abstractor computes abstract states as,
\begin{equation}
    A_i \gets \mathrm{FeedForward}\paren{\sum_{j} \iprod{\phi(x_i)}{\psi(x_j)} s_j}.
\end{equation}

The following result shows that a 1-layer Abstractor can approximate arbitrary functions of each object's relations with the other objects. This result follows from the analysis of function classes of inner products of neural networks in~\citep{TODO:POSTtoARXIV}.
\begin{result}\label{result:abstractor_univ_approximation}
    Let $r: \calX \times \calX \to \reals^{d_r}$ be any relation function. Then, there exists a choice of symbols $s_1, \ldots, s_\m$ and parameters of the feed-forward network such that $A_i$ approximates an arbitrary function of $\paren{r(x_i, x_j)}_{j=1}^\m$.
\end{result}

As discussed in the next section, Abstractor modules can be composed in a broader architecture to compute higher-order relations. The following result states that a composition of $k$ single-layer Abstractors can approximate arbitrary $k$th-order relational functions.

\begin{result}\label{result:abstractor_composition}
    A composition of $k$ single-layer Abstractors is able to compute arbitrary $k$th-order relational functions.
\end{result}

Finally, we present a robustness result which states that relational cross-attention is able to encodes relations robustly via redundancy in the symbols.

\begin{result}\label{result:abstractor_robustness}
    Suppose that symbolic message-passing is used to transform a sequence of $m$ symbols, each of dimension $d$. If a fraction $\epsilon$ of the entries of the transformed symbols are arbitrarily corrupted, the relations can be exactly recovered by a linear program as long as $\sqrt{m/(1-\epsilon)^2d}$ is sufficiently small.
\end{result}

The results of~\citep{model_repair} make the robustness properties precise. We investigate different forms of robustness empirically in the experiments section.

\section{Abstractor architectures}\label{sec:abstractor_architectures}

% Similar to a Transformer Encoder, an Abstractor is a module that takes a sequence of objects $X = (x_1, \ldots, x_\m)$ as input and produces a processed representation of the input as another sequence of objects, $A = (A_1, \ldots, A_\m)$. In contrast to the encoder states $E = \paren{E_1, \ldots, E_\m}$, which represent an entangled mixture of object-level attributes and relational information, the abstract states $A$ represent only relational information of the input, abstracted away from the object-level attributes. An Abstractor is a module whose purpose is to process purely relational information. It can be integrated into a broader transformer-based architecture in several different ways, depending on the target task.

Whereas a Transformer Encoder performs ``general-purpose'' processing, extracting representations of both object-level and relational information, an Abstractor module is more specialized and produces more enriched relational representations. An Abstractor module can be integrated into a broader transformer-based architecture, for enhanced relational processing.

To facilitate the discussion of different architectures, we distinguish between two types of tasks. In a \textit{purely relational} prediction task, there exists a sufficient statistic of the input which is purely relational and encodes all the information that is relevant for predicting the target. The experiments of~\citep{esbn,kerg2022neural} are examples of purely relational discriminative tasks. We consider discriminative relational tasks in~\Cref{ssec:experiments_discriminative}. An example of a purely relational \textit{sequence-to-sequence} task is the object-sorting task described in~\Cref{ssec:experiments_object_sorting}.
% To predict the `argsort' of a sequence of objects, the pairwise $\prec$ relation is sufficient.
Many real-world tasks, however, are not purely relational. In a \textit{partially-relational} prediction task, the relational information is important, but is not sufficient for predicting the target. The math problem-solving experiments in~\Cref{ssec:experiments_math} are partially-relational. Natural language understanding is also an example of a partially-relational task.

\begin{figure}
    \centering
    \includegraphics[width=.8\textwidth]{figures/abstractor_architectures.pdf}
    \caption{Examples of Abstractor-based model architectures.}\label{fig:abstractor_architectures}
    \vskip-10pt
\end{figure}


The way that an Abstractor module is integrated into a broader model architecture should be informed by the underlying prediction task.~\Cref{fig:abstractor_architectures} depicts several Abstractor architectures each with different configurations. Architecture (a) depicts a configuration in which the Abstractor processes the relational features in the input, and the decoder attends to the abstract states $A$. Architecture (b) depicts a configuration in which the input objects are first processed by an Encoder, followed by an Abstractor for relational processing, and the decoder again attends to the abstract states. These architectures would be appropriate for purely relational tasks, since the decoder attends only to the relational representations in the abstract states. Architectures (c) and (d), in which information can also pass directly from the encoder to the decoder, would be more appropriate for more general tasks which are only partially relational. For example, in architecture (c), the model branches into two parallel processing streams in which an Encoder performs general-purpose processing and an Abstractor performs more specialized processing of relational information.
The decoder attends to \textit{both} the encoder states and the abstract states.
% We refer to architectures which combine sensory processing with abstract relational processing as \textit{sensory-connected} architectures.
These architectures use the ``multi-attention decoder'' described in~\Cref{sec:multi_attn_decoder}.
Finally, architecture (e) depicts a \textit{composition} of Abstractors, wherein the abstract states produced by one Abstractor module are used as input to another Abstractor. This results in computing ``higher-order'' relational information (i.e., relations on relations).
% This is made formal in~\Cref{ssec:function_classes_preview}. % we cut out the abstractor theory section. hopefully we can include it in a final version?
% We leave empirical evaluation of this architecture to future work. 
% 
%\newcommand{\MLP}{\text{MLP}}
%\newcommand{\FeedForward}{\text{FeedForward}}
%\newcommand{\Softmax}{\text{Softmax}}
%\newcommand{\reals}{\mathbb{R}}
\def\m{m}


\section{The Abstractor Framework}
\label{sec:abstractor_framework}

At a high level, the primary function of an abstractor is to compute abstract relational features of its
inputs.\footnote{In this paper, we will tend to use the name `Abstractor' to refer to both the module, the framework, and models which contain the Abstractor module as a main component.} That is, given a set or sequence of input objects $o_1, \ldots, o_\m$, the relational abstractor learns to model a relation $r(\cdot, \cdot)$ and computes a function on
the set of pairwise relations between objects ${\{ r(o_i, o_j) \}}_{ij}$. The relations and the computations on them are learned to carry out a specific prediction task. This learning is often end-to-end, but, crucially, the Abstractor framework naturally supports modular learning.
\subsection{Relational symbolic message-passing}
\label{ssec:message_passing}

At the core of abstractors is an operation we refer to as \textit{relational symbolic message-passing}.
The input to this operation is a relation tensor $R = \left[r(o_i, o_j)\right]_{i,j=1}^\m$, where $r(o_i, o_j) \in \mathbb{R}^{d_r}$ is a vector describing the relation between object $o_i$ and object $o_j$. We will come back to how an abstractor models pairwise relations and computes the relation tensor in the next subsection.

The starting point of symbolic message-passing is a set of learnable symbols $s_1, \ldots, s_\m \in \mathbb{R}^{d_s}$, where the hyperparameter $d_s$ is the dimension of the symbolic vectors. We call these parameters \textit{symbols} because each of them references (or ``is bound to'') a particular object, but they are independent of the values of these objects. That is, the $i$th symbol references the $i$th object, but the value of $s_i$ is independent of the value of $o_i$. The use of these learned, input-independent symbols is how symbolic message-passing achieves its abstraction.

In relational symbolic message-passing, we perform message-passing on these learned symbolic parameters according to the relation tensor $R$. In general, this message-passing operation can be described as a set-valued function of the form
\begin{equation}
    \label{eq:symbolic_message_passing}
    s_i \leftarrow \text{Update}\Big( s_i, \ \left\{ \left(R[i,j], s_j\right)\right\}_{j\in[m]}\Big), \quad i = 1, \ldots, m.
\end{equation}

That is, the value of the $i$th symbol is updated as a function of the set of tuples $(R[i,j], s_j)$ of the relations with all other objects and the symbols of these objects. The symbols $s_j$ are naturally viewed as values on the nodes of a graph, and the relations $R[i,j]$ are naturally viewed as weights on the edges. A simple but important special case of this is linear message-passing
\begin{equation}
    \label{eq:linear_symbolic_mp}
    s_i \leftarrow \sum_{j=1}^{m} R[i,j] s_j, \quad i=1, \ldots, m
\end{equation}

In the above, if $d_r > 1$, the operation should be read as
\begin{equation*}
    R[i,j] s_j = \left( R[i,j,1] s_j, \ldots, R[i,j, d_r] s_j \right) \in \reals^{d_s \times d_r},
\end{equation*}

where $d_r$ is the dimension of the relation. That is, the result is concatenated.

Following message-passing, each updated symbol $s_i$ can be passed through a feedforward neural network $\phi: \reals^{d_s \times d_r} \rightarrow \reals^{d_a}$ to compute a non-linear function of the output. This also controls the dimension of the symbols so that it doesn't grow by a factor of $H$ with each layer (e.g., take $d_a = d_s$). Empirically, a residual connection and layer normalization may be useful.

This message-passing operation can be repeated multiple times to iteratively update the symbolic vectors.  The output of relational symbolic message-passing is the set of symbols $\bm{A}$ at the end of this sequence of message-passing operations. This is summarized in~\Cref{alg:symbolic_mp}.

\begin{algorithm}[ht!]
	\caption{Symbolic Message-Passing}\label{alg:symbolic_mp}
	\SetKwInOut{Input}{Input}
	\SetKwInOut{Output}{Output}
	\SetKwInOut{LearnableParams}{Learnable parameters{\ }}
	\SetKwInOut{HyperParams}{Hyperparameters}

	\Input{Relation tensor: \(R \in \mathbb{R}^{\m \times \m \times d_r}\)}
	\HyperParams{\(L,\ d_s, d_a\), hyperparameters of feedforward networks}
	\LearnableParams{symbols \(\bm{S} = (s_1, \ldots, s_\m) \in \reals^{d_s \times \m}\), feedforward networks \(\phi^{(1)}, \ldots, \phi^{(L)}\)}
	\Output{Abstracted sequence: \(\bm{A} = (a_1, \ldots, a _\m) \in \reals^{d_a \times \m}\)}
	\vspace{1em}

	\((a_1, \ldots, a_\m) \gets (s_1, \ldots, s_\m)\)

	\For{\(l \gets 1\) \KwTo \(L\)}{
		\(a_i \gets \sum_{j=1}^{n} R[i,j] a_j, \quad i = 1, \ldots, \m\)

		\(a_i \gets \phi^{(l)}(a_i), \quad i = 1, \ldots, \m\)
	}
\end{algorithm}


\subsection{Multi-head relations and relational cross-attention}

Next, we turn our attention to how the Abstractor models pairwise relations and computes the relation tensor $R \in \mathbb{R}^{\m \times \m \times d_r}$.

The inner product operation is a natural way to capture notions of relations and similarity. In Euclidean space, inner products capture the geometric alignment between vectors. Similarly, for arbitrary objects with vector representations, inner products between these vector representations can capture relations between these objects.

We model pairwise relations as inner products between appropriately encoded (or `filtered') object representations. In particular, we model the pairwise relation function $r(\cdot, \cdot) \in \mathbb{R}^{d_r}$ in terms $d_r$ learnable `left encoders' $\phi_1, \ldots, \phi_{d_r}$, and $d_r$ `right encoders' $\psi_1, \ldots, \psi_{d_r}$,
% \begin{equation}\label{eq:multi_head_rel}
%     r(x,y) = \left(\langle \phi_1(x), \psi_1(y) \rangle, \langle \phi_2(x), \psi_2(y) \rangle, \ldots, \langle \phi_{d_r}(x), \psi_{d_r}(y) \rangle \right)^\top \in \mathbb{R}^{d_r}.
% \end{equation}

\begin{equation}\label{eq:multi_head_rel}
    r(x,y) = \begin{pmatrix}\langle \phi_1(x), \psi_1(y) \rangle \\ \vdots \\ \langle \phi_{d_r}(x), \psi_{d_r}(y) \rangle \end{pmatrix} \in \mathbb{R}^{d_r}.
\end{equation}


In general, $\phi_i, \psi_j$ can be any learnable maps. These transformations can be thought of as \textit{relational filters}. They extract a particular attribute of the objects such that an inner product of the transformed objects indicates the alignment or similarity along this attribute. Having several different filters allows for modeling rich multi-dimensional relations. This is one notable advantage of this formulation over the CoRelNet model \citep{kerg2022neural}, which processes a \textit{1-dimensional} similarity matrix as input to a multi-layer perceptron. In the next section, we analyze the class of functions that the multi-head relation module can model.

In order to promote weight sharing, we focus our attention to inner product relations of the form

\begin{equation}\label{eq:inner_prod_rel_weight_sharing}
    r(x,y) = \begin{pmatrix} \left\langle W_1^{(1)}\phi(x), W_2^{(1)} \phi(y) \right\rangle \\  \vdots \\ \left\langle W_1^{(d_r)}\phi(x), W_2^{(d_r)} \phi(y) \right\rangle \end{pmatrix} \in \mathbb{R}^{d_r},
\end{equation}

\noindent where $\phi$ is a common non-linear map, and $W_1^{(i)}, W_2^{(i)}$ are projection matrices for each dimension of the relation. For general functions $\phi$, this class of functions is no smaller than the one above (e.g., take $\phi$ to be the concatenation of $\phi_1, \ldots, \phi_{d_r}, \psi_1, \ldots, \psi_{d_r}$ and $W_1^{(i)}, W_2^{(i)}$ to be the projection matrices which extract the appropriate components), but does enable greater weight sharing.

We refer to this operation as \textit{Multi-Head Relation} (\Cref{alg:multiheadrelation}). In our implementation, computation of the inner product is done efficiently with Einstein summation. Also, we add a hyperparameter to control whether the relations are modeled as symmetric or asymmetric (as in the description above). If the relations are to be modeled as symmetric, we set $W_1^{(i)} = W_2^{(i)}$. For certain tasks where relations may be naturally symmetric, this may be a useful inductive bias which improves sample-efficiency (e.g., see the discussion in~\cite{kerg2022neural}).

\begin{algorithm}[ht!]
	\caption{Multi-Head Relation (MHR) module}\label{alg:multiheadrelation}
	\SetKwInOut{Input}{Input}
	\SetKwInOut{Output}{Output}
	\SetKwInOut{LearnableParams}{Learnable parameters{\ }}
	\SetKwInOut{HyperParams}{Hyperparameters}

	\Input{sequence of objects: $\bm{X} = (x_1, \ldots, x_\m) \in \reals^{d}$ }
	\HyperParams{Dimension of relation $d_r$, Projection dimension $d_p$, dimension of embedding $d_e$}
	\LearnableParams{projection matrices $W_1^{(i)}, W_2^{(i)} \in \reals^{d_p \times d_e}, \ i=1, \ldots, d_r$, embedder network $\phi: \reals^d \to \reals^{d_e}$}
	\Output{Relation tensor $R \in \reals^{\m \times \m \times d_r}$}
	\vspace{1em}

	\For{\(i,j \gets 1\) \KwTo \(\m\)}{
        \For{$k \gets 1 $ \KwTo $d_r$}{
            $R[i, j, k] \gets \langle W_1^{(k)} \phi(x_i), W_2^{(k)} \phi(x_j)\rangle$
        }
	}
\end{algorithm}

\subsection{The Abstractor module: Putting it all together}

The above two sections provide a complete description of relational symbolic message-passing and computing relations via multi-head relation modules. These are the two main components of the Abstractor module.

The initial abstract state is $\bm{S} = (s_1, \ldots, s_\m)$ with abstract symbols $s_j$ that are task-dependent but input-independent, trainable using backpropagation. The multi-head relation module learns relations among the input objects and uses those relations to transform the abstract state. Importantly, each `head' of the multi-head relation module encode learned relations and attributes which can be reused across tasks.

Only relational information, computed through inner products, is used to transform the abstract variables; no information about the individual object representations themselves is directly accessed by the abstract side. This enables greater out-of-distribution generalization ability since it allows for the representations of the objects to change as long as the transformed inner products are approximately preserved (see~\Cref{sec:experiments} for experiments exploring this). This is a crucial component of the idea of the relational bottleneck.

\Cref{alg:abstractor} provides an algorithmic description of the archetypical abstractor.

\begin{algorithm}[ht!]
	\caption{Abstractor}\label{alg:abstractor}
	\SetKwInOut{Input}{Input}
	\SetKwInOut{Output}{Output}
	\SetKwInOut{LearnableParams}{Learnable parameters}
	\SetKwInOut{HyperParams}{Hyperparameters}

	\Input{sequence of objects: $\bm{X} = (x_1, \ldots, x_\m) \in \reals^{d}$ }
	\HyperParams{\# of layers \(L\), dim of symbols \(d_s\), dim of abstract objects \(d_a\), hyperparameters of MHR modules, activation function for relation tensor \(\sigma_{\mathrm{rel}}\), hyperparameters of feedforward networks.}
	\LearnableParams{symbols \(\bm{S} = (s_1, \ldots, s_\m) \in \reals^{d_s \times \m}\), feedforward networks \(\phi^{(1)}, \ldots, \phi^{(L)}\), parameters of MHR modules.}
	\Output{Abstracted sequence: \(\bm{A} = (a_1, \ldots, a _\m) \in \reals^{d_a \times \m}\)}
	\vspace{1em}

	\((a_1, \ldots, a_\m) \gets (s_1, \ldots, s_\m)\)

	\For{\(l \gets 1\) \KwTo \(L\)}{
        \(R \gets \text{MultiHeadRelation}^{(l)}(\bm{X}) \)

        \(R \gets \sigma_{\mathrm{rel}}(R)\)

		\(a_i \gets \sum_{j=1}^{n} R[i,j] a_j, \quad i = 1, \ldots, \m\)

		\(a_i \gets \phi^{(l)}(a_i), \quad i = 1, \ldots, \m\)
	}
\end{algorithm}

Note that the relation tensor output by the multi-head relation module is processed with an activation function $\sigma_{\mathrm{rel}}$. Depending on the task, one good choice for this is the softmax activation function. This normalizes the message-passing operation such that each abstract symbol is updated as a \textit{convex combination} of the other symbols based on the relation tensor. It is important to note that this causes the computed relation between two objects to depend also on the relations with other objects (i.e., $R[i,j]$ depends not only on $x_i, x_j$, but on the full object sequence). Thus, softmax computes the relation between two objects \textit{relative} to the relations with all objects. Depending on the application, this may be a very useful inductive bias or a harmful one. Alternatively, we may apply an activation function $\sigma_{\mathrm{rel}}$ independently for each entry in the relation tensor (e.g., linear, relu, sigmoid, tanh, etc.)

Finally, we remark that this operation is closely related to the multi-head attention operation of transformers. In fact, computing the relation tensor and performing symbolic message-passing can be achieved via a multi-head attention operation of the form
\begin{equation}\label{eq:relation_crossattention}
    \text{RelationalCrossAttention}\left(E, S\right) \equiv \text{Attention}\left( Q \gets E, K \gets E, V \gets S \right),
\end{equation}
where $E$ are the input objects (or some processed-encoding of them), and $S = (s_1, \ldots, s_\m)$ are the symbolic variables. We refer to this operation as \textit{relational cross-attention}. This is in contrast to standard (encoder-decoder) cross-attention, which takes the form $\text{Attention}\left( Q \gets D, K \gets E, V \gets E \right)$.

For completeness, ~\Cref{alg:relational_abstractor} gives an algorithmic description of the Abstractor module, cast in terms of transformer-based attention mechanisms. Note that we have added a self-attention operation performed on the abstract symbols. This is merely to show this as an option. It may be useful for some tasks, but, unlike the rest of Abstractor, it not intuitive what this might be computing.

\begin{algorithm}[th!]
    \caption{Abstractor (cast in terms of transformer operations)}\label{alg:relational_abstractor}
    \SetKwFor{For}{for}{do}{end}
    \SetKwInOut{Input}{Input}
    \SetKwInOut{Output}{Output}
    \SetKwInOut{LearnableParams}{Learnable parameters{\ }}
    \SetKwInOut{HyperParams}{Hyperparameters}

    \Input{sequence of objects: $\bm{X} = (x_1, \ldots, x_\m) \in \reals^{d}$ }
	\HyperParams{\# of layers \(L\), dim of symbols \(d_s\), dim of abstract objects \(d_a\), hyperparameters of attention modules, hyperparameters of feedforward networks.}
	\LearnableParams{symbols \(\bm{S} = (s_1, \ldots, s_\m) \in \reals^{d_s \times \m}\), feedforward networks \(\phi^{(1)}, \ldots, \phi^{(L)}\), parameters of attention modules.}
	\Output{Abstracted sequence: \(\bm{A} = (a_1, \ldots, a _\m) \in \reals^{d_a \times \m}\)}

    \vspace{1em}

    $\bm{A} \gets \bm{S}$

    \For{$l \gets 1$ \KwTo $L$}{
        $\bm{A} \gets \text{SelfAttention}^{(l)}(\bm{A})$\;
        $\bm{A} \gets \text{RelationalAttention}^{(l)}(\bm{X}, \bm{S})$\;
        $\bm{A} \gets \phi^{(l)}(\bm{A})$\;
        }
\end{algorithm}

\subsection{Relational learning using transformers}

The Abstractor module fits naturally into Transformer models. Hence, we position the Abstractor framework as an extension of Transformers. In particular, combining an Abstractor with Transformer Encoder and Decoder modules forms a powerful sequence model with enhanced abilities for relational reasoning and abstraction.

In our extended framework, processing occurs in encoder/decoder modules that handle particular types of information,
separated from modules for ``abstract inference.'' The encoders/decoders and the abstractor modules communicate
through cross-attention mechanisms that couple abstract states with specific information in encoder/decoder modules.  The abstract layers are composable to include a hierarchy of abstract modules in which higher order relations are learned from lower level relations, analogous to how convolutional layers are composed in deep neural networks.

%\begin{figure}[t]
    \begin{wrapfigure}{R}{0.50\textwidth}
        \vspace{-3mm}
        \begin{center}
        \begin{tabular}{c}
            \hskip5pt\includegraphics[width=.45\textwidth]{figures/algorithm-diagram3-crop}
        \end{tabular}
        \caption{\footnotesize Algorithmic framework integrating transformers and relational learning, 
        implementing a form of the ``relational bottleneck.''}
        \label{fig:algo}
        \end{center}
        \vskip-.15in
    \end{wrapfigure}
    %\end{figure}
The architecture has three types of states: encoder states $E$, decoder states $D$, and abstract states $A$. The encoder states are vectors that represent domain-specific information (e.g., sensory or motor), which are often successfully modeled by standard deep learning frameworks, including standard transformers. The abstract states $A$ are vectors that are learned and processed using symbolic message-passing based on the relations between the encoder states. In particular, the encoder states are separated from the abstract states by a ``relational bottleneck'' that only allows information about relations (that is, inner-products) between encoder states to influence the learning and transformation of abstract states.

This ability to integrate with transformers and process domain-specific information modeled by an Encoder gives the Abstractor framework greater flexibility compared to existing relational models like ESBN.
%If sensory information is required by the decoder, this can be passed to the decoder states $D$ by standard cross-attention mechanisms of transformers.


%
%\subsection{Example}
%\label{ssec:set}
%
%We use the game SET to illustrate relational cross-attention.
%% JDC: COULD REFERENCE WIKIPEDIA PAGE FOR SET:  https://en.wikipedia.org/wiki/Set_(card_game)
%SET is a relatively straightforward but challenging cognitive task that engages reasoning faculties in a deliberative
%, attentionally directed manner, requiring several levels of abstraction over sensory embeddings. Players are
%presented with 12 cards, each of which contains figures that vary along four dimensions (color, number, pattern, and
%shape; see Figure \ref{fig_set}a) and they must find subsets of three cards which obey a deceptively simple rule: along each dimension, all cards in a set must either have the same or unique values (e.g., in Figure \ref{fig_set}, cards with two solid blue/purple diamonds, two striped blue squiggles, and two open blue oblongs: same color, same number, different patterns, different shapes).
%
%Algorithmically, task performance can be described as follows. The visual arrangement of cards is processed into a set of encoder states $E$ by standard deep learning mechanisms. The abstractor, starting in some initial abstract state $A$, transforms the state by the evaluation of attention heads that extract relations between the cards, each head giving a relation between them in a learned attribute.
%
%\def\redcard{\colorbox{red!30}{R}\hskip.2em}
%\def\bluecard{\colorbox{blue!30}{B}\hskip.2em}
%\def\greencard{\colorbox{green!50}{G}\hskip.2em}
%\def\onecard{\fbox{\hskip1pt 1\hskip1pt}\hskip.2em}
%\def\twocard{\fbox{\hskip1pt 2\hskip1pt}\hskip.2em}
%\def\threecard{\fbox{\hskip1pt 3\hskip1pt}\hskip.2em}
%
%\begin{figure}[t]
%\begin{center}
%\begin{tabular}{ccc}
%\includegraphics[width=.23\textwidth]{figures/set_example}
%& &\\[-1.15in]
%&\hskip10pt\ &\renewcommand{\arraystretch}{1.4}
%\begin{small}
%\begin{tabular}{|c|c|c|}
%%\hline
%\multicolumn{3}{c}{Attributes of encoder state $E$} \\
%\hline
%\multicolumn{1}{|c}{\redcard \redcard \redcard} & \multicolumn{1}{|c}{ \redcard\bluecard\greencard} &\multicolumn{1}{|c|}{\redcard\redcard\bluecard} \\
%\hline
%\hline
%%$\frac{1}{3}(s_1+s_2+s_3)$ & $\frac{1}{2}(s_2+s_3)$ & $s_3$ \\
%%$\frac{1}{3}(s_1+s_2+s_3)$ & $\frac{1}{2}(s_1+s_3)$ & $s_3$ \\
%%$\frac{1}{3}(s_1+s_2+s_3)$ & $\frac{1}{2}(s_1+s_2)$ & $\frac{1}{2}(s_1+s_2)$ \\
%$\frac{1}{3}(s_1+s_2+s_3)$ & $s_1$ & $\frac{1}{2}(s_1+s_2)$ \\
%$\frac{1}{3}(s_1+s_2+s_3)$ & $s_2$ & $\frac{1}{2}(s_1+s_2)$ \\
%$\frac{1}{3}(s_1+s_2+s_3)$ & $s_3$ & $s_3$ \\
%\hline
%\multicolumn{3}{c}{Transformed abstract symbol $A$}
%%\hline
%\end{tabular}
%\end{small}
%%& \\[-1.45in]
%%&&\includegraphics[width=.23\textwidth]{ppo-results/epoch_14_decision_boundaries} \vspace{-2mm} \\
%\\[10pt]
%\scriptsize (a) SET game && \scriptsize (b) Example of symbols/attention
%\end{tabular}
%\vspace{-1mm}
%\end{center}
%\caption{Illustration of mechanism used in abstractor layers with relational cross attention using the game of SET (see text for description).
%%To provide a basic working of example of abstraction in our framework,
%As an example, consider an initial abstract symbol $S = (s_1,s_2,s_3)$ for three cards, and the color attribute.
%%, for concreteness.
%Suppose a relation is learned such that  $\langle W_Q E_i, W_K E_j\rangle$ is small if cards $i$ and $j$ have different color, and is large if they are the same color.  Then relational cross-attention transforms the initial abstract symbol $S$ as shown in the above table. A multilayer perceptron can learn to discriminate between these three cases. Once learned, the symbol $A$ then can be used to represent abstract ``same/different'' relations for the sequence of inputs.}
%\label{fig_set}
%\end{figure}
%


\subsection{Configuring abstractors for different tasks}
\def\module#1{\mbox{\small\texttt{#1}}}

Abstractors can be used to approach a variety of relational learning tasks. In the case of classification
or regression, the default architecture would be
$$\module{Encoder} \rightarrow \module{Abstractor}$$
and the discriminant or regression function is computed as $f(A)$, where $A$ is the final abstract states.
For relational sequence-to-sequence tasks, the default architecture is
$$\module{Encoder} \rightarrow \module{Abstractor} \rightarrow \module{Decoder}$$

In a ``fully relational'' task, the decoder only attends to the abstractor, and therefore only uses relational information from the input. Fully relational tasks are those which can be solved using only relational information, without any information about the individual objects. An example of a fully relational task is sorting objects; we give experimental details for this example in Section~\ref{sec:experiments}.

In a ``partially-relational'' task, the relational information is crucial, but information about individual objects is also important. Here, we propose an architecture in which the decoder attends to both the abstractor and encoder modules. This can be done by either concatenating the encoder and abstractor states (i.e.: attend to $\text{concat}(E, A)$) or to iteratively cross-attend to the Encoder and Abstractor. We call this a ``sensory-connected'' Abstractor.

% JDC: IT MIGHT BE GOOD, IF POSSIBLE TO INCLUDE AN "INLINE SCHEMATIC" FOR THIS CASE AS WELL, WITH A "BYPASS" ARROW
% (RESNET STYLE) FROM ENCODER TO DECODER (THOUGH I DON'T KNOW HOW TO DO THIS IN TEX!), TO MAKE IT VISIBLY CLEAR THAT
% THE FRAMEWORK IS FLEXIBLE.
This provides an extension of general sequence-to-sequence models with transformers. In this paper, to highlight the capabilities of the framework, we focus our experiments on fully relational abstractors. However, partially relational abstractors are likely to be necessary for more realistic tasks. We hypothesize that a sensory-connected Abstractor model would yield benefits on language tasks.

We note that learning higher order relations is made possible by composing
abstractors, as in the architecture
\begin{equation*}\module{Encoder} \rightarrow \module{Abstractor} \rightarrow \module{Abstractor} \rightarrow \module{Decoder}.
\end{equation*}
Since a one-layer abstractor is able to compute a large class of functions on relations, chaining together abstractors allows the computation of relations on relations (higher-order relations). We formalize these comments in Section~\ref{sec:function_spaces}.


% \section{Function classes}
\label{ssec:function_classes}

In the supplementary material, we discuss the class of relational functions computable by the symbolic message-passing operation in relational abstractors, as well as the robustness of these operations. In the process, we characterize the class of relational functions realizable by inner product relational neural networks, which may be of independent interest. These results are important for appreciating the expressivity of relations and symbolic message passing, but are more technical and we therefore gather the results in the supplement. A summary of the results follows.


\def\rdot{\bigcdot}
\def\F{{\mathfrak{F}}}
\def\MLP{\text{MLP}}

Our first result is a universal approximation result for inner product relations. This is useful when characterizing the class of functions computable by abstractors, but is also of interest more generally for relational machine learning.
We would like to learn a relation function \(R: \mathcal{X} \times \mathcal{X} \to \reals^{d_r}\) which maps pairs of objects in \(\mathcal{X}\) to a \(d_r\)-dimensional vector describing the relation between these objects. We model this relation function as a vector of inner products between transformations of the objects' representations:
\begin{equation}
	\label{eq:inner_product_relations}
	R(x, y) = \begin{pmatrix}\langle \phi_{1}(x), \phi_{1}(y) \rangle \\ \vdots \\ \langle \phi_{d_r}(x), \phi_{d_r}(y) \rangle \end{pmatrix},
\end{equation}
where \(\phi_{1}, \ldots, \phi_{d_r}\) are learnable transformations corresponding to each dimension of the relation. These transformations can be thought of as \textit{relational filters}. They extract a particular attribute of the objects such that an inner product of the transformed objects indicates the alignment or similarity along this attribute. Having several different filters allows for modeling rich multi-dimensional relations. This is one notable advantage of this formulation over the CoRelNet model \citep{kerg2022neural}, which processes a relation matrix as input to a multi-layer perceptron.

Our first result characterizes the class of inner product relations computable by \eqref{eq:inner_product_relations} when \(\phi_{1}, \ldots, \phi_{d_r}\) are feedforward networks. We make use of Mercer's theorem and universal approximation properties of feedforward networks to obtain a universal approximation result for inner product relational neural networks. This is contained in Theorem~\ref{theorem:function_class_inner_product_relnn}.

Additionally, we can consider inner products of the form
\begin{equation}
	\label{eq:nonsymmetric_inner_product_relnn}
	\langle W_k^{(1)} \phi(x_i), W_k^{(2)} \phi(x_j) \rangle,
\end{equation}
\noindent where the linear projections for the first and second entities may be different, in order to achieve non-symmetric relation functions. This yields a strictly larger class of relation functions than in Theorem \ref{theorem:function_class_inner_product_relnn}.

\subsection{Class of relational functions computable by symbolic message-passing}
\label{ssec:function_class_symbolic_mp}
For the purposes of this analysis, the algorithmic description of symbolic message-passing is presented in \Cref{alg:symbolic_mp}. It is slightly simpler than the algorithmic description of the full relational  abstractor in \Cref{alg:relational_abstractor}---the primary difference is that we omit self-attention between symbolic states.

\begin{algorithm}[ht!]
	\caption{Symbolic Message-Passing}\label{alg:symbolic_mp}
	\SetKwInOut{Input}{Input}
	\SetKwInOut{Output}{Output}
	\SetKwInOut{LearnableParams}{Learnable parameters{\ }}
	\SetKwInOut{HyperParams}{Hyperparameters}

	\Input{Relation tensor: \(R \in \mathbb{R}^{n \times \m \times d_r}\)}
	\HyperParams{\(L\) (number of steps/layers), hyperparameters of feedforward networks}
	\LearnableParams{symbols \(\boldsymbol{s} = (s_1, \ldots, s_\m) \in \reals^{d_s \times \m}\), feedforward neural networks \(\phi^{(1)}, \ldots, \phi^{(L)}\)}
	\Output{Abstracted sequence: \(\boldsymbol{a} = (a_1, \ldots, a _\m) \in \reals^{d_a \times \m}\)}
	\vspace{1em}

	\(A \gets S\)

	\For{\(l \gets 1\) \KwTo \(L\)}{
		\(a_i \gets \sum_{j=1}^{n} R[i,j] a_j, \quad i = 1, \ldots, n\)

		\(a_i \gets \phi^{(l)}(a_i), \quad i = 1, \ldots, n\)
	}
\end{algorithm}



From equation \eqref{eq:linear_symbolic_mp}, the symbolic message-passing operation is clearly bijective as a function on the input relation tensor \(R\), for an appropriate choice of the symbol parameters \(S = (s_1,\ldots, s_\m)\). For example, choosing \(S = I_{\m \times \m}\) (i.e.: the \(i\)th symbolic vector is the indicator \(\m\)-vector with a \(1\) in the \(i\)th position, \(s_i = e_i\)) reproduces the relation tensor after one message-passing operation:
\begin{equation*}
	s_i  \leftarrow  \sum_{j=1}^{n} R[i,j] e_j = \begin{pmatrix}R[i,1] \\ R[i,2] \\ \vdots \\ R[i,n]\end{pmatrix}.
\end{equation*}
More generally, one linear step of symbolic message-passing yields updated symbolic vectors such that \(s_i'\) is a linear function of the vector of all objects' relations with object \(i\):
\begin{equation*}
	s_i \leftarrow = S \begin{pmatrix}R[i,1] \\ \vdots \\ R[i,n]\end{pmatrix}.
\end{equation*}
Following the linear step in symbolic message-passing, each updated symbolic state is transformed via a neural network. Hence, the \(i\)th abstracted value after symbolic message-passing is given by
\begin{equation*}
	a_i = \phi\left(S r_i \right),
\end{equation*}
where \(\phi\) is a neural network, and \(R_i\) is the vector of object \(i\)'s relations with every other object, \(R_i = \begin{pmatrix}R[i,1] & \cdots & R[i,n]\end{pmatrix}^\top\). Hence, \(a_i\) summarizes all the information about object \(i\)'s relations to all other objects. We summarize this discussion by the following
lemma, which follows from universal approximation properties of feed-forward networks.

\begin{lemma}
	\label{lemma:function_class_1_step_symbolic_mp}
	A one-step symbolic message-passing operation (in \Cref{alg:symbolic_mp}) can compute arbitrary functions of a each object's relations with other objects in the input sequence. That is, there exists a choice of symbols \(s_1, \ldots, s _\m\) and parameters of the feed-forward network such that \(a_i\) computes an arbitrary function of object \(i\)'s relations, \(r_i = \begin{pmatrix}R[i,1] & R[i,2] & \cdots & R[i,n]\end{pmatrix}^\top\).
\end{lemma}


Thus, the abstracted sequence after a single step of symbolic message-passing has the form
\begin{equation}
	\label{eq:abstracted_seq_1_layer_abstractor}
	A^{(1)} = (a_1^{(1)}, \ldots, a_\m^{(1)}) = \left(\phi(r_1), \phi(r_2), \ldots, \phi(r_\m)\right),
\end{equation}
where \(\phi\) is an arbitrary learnable function shared by all abstracted objects, and \(r_i\) is the vector of object \(i\)'s relations with every other object.

That is, \(a_i^{(1)}\) summarizes object \(i\)'s relations with other objects. With further symbolic message-passing operations, the \(i\)th abstracted vector can be made to represent information about other relations, not necessarily involving the \(i\)th object. For example, at the second layer, the abstracted vectors take the form
\begin{equation}
	a_i^{(2)} = \phi^{(2)} \left( \sum_{j=1}^{n} R[i,j] a_j^{(1)} \right) = \phi^{(2)} \left( \sum_{j=1}^{n} R[i,j] \phi^{(1)}(R_j) \right).
\end{equation}

We conclude this subsection by remarking that the above analysis concerns \Cref{alg:symbolic_mp}---a simplified version of the relational  abstractor. In particular, while it captures the effects of relational cross-attention, it does not include self-attention on the abstract symbols. The analysis indicates that we should expect the function class generated by a relational  abstractor module in \Cref{alg:relational_abstractor} to be no smaller than that of the simple symbolic message-passing in \Cref{alg:symbolic_mp}.


\subsection{Composing  abstractors to compute relations on relations}
\label{ssec:compsing_abstractors}

As described in \Cref{sec:abstractors_as_transformer_modules}, the abstactor framework supports composing  abstractors in the form
\begin{equation*}
	\texttt{Encoder} \to \texttt{Abstractor} \to \cdots \to \texttt{Abstractor} \to \texttt{Output}
\end{equation*}
Here, we analyze the function class generated by a composition of several abstractors. We make the simplifying assumption that each single layer abstractor takes the simplified form of the symbolic message-passing operation in \Cref{alg:symbolic_mp}. This corresponds to omitting the self-attention operation in \Cref{alg:relational_abstractor} while maintaining the relational cross-attention with the sequence of output vectors at the previous  abstractor.

We saw in the previous section that a one-layer abstractor is able to compute arbitrary functions  of each object's relations. Observe that the output sequence of abstracted objects is a sequence of `relational vectors'. That is, objects which summarize relational information. Hence, chaining together a sequence of  abstractors allows the computation of relations on relations.

% TODO: formalize or refine presentation of result
\begin{lemma}
	\label{lemma:function_class_composed_abstractors}
	A chain of \(k\) single-layer  abstractors is able to compute arbitrary ``\(k\)th order relational functions'' in the sense of the proof below.
\end{lemma}
\begin{proof}[Proof sketch]
	In \Cref{ssec:function_class_symbolic_mp} we characterized the output of a 1-layer abstractor as
	\begin{equation*}
		A^{(1)} = (a_1^{(1)}, \ldots, a_\m^{(1)}) = \left(\phi^{(1)}(r_1), \phi^{(1)}(r_2), \ldots, \phi^{(1)}(r_\m)\right),
	\end{equation*}
	Note that we will now use the superscript to denote the abstractor in the chain rather than the layer depth in a single  abstractor, as all  abstractors have a depth of one.

	Let the second abstractor's symbols be denoted by \(S^{(2)} = (s_1^{(2)}, \ldots, s_\m^{(2)})\). Then,
	\begin{equation*}
		a_i^{(2)} = S^{(2)} \begin{bmatrix}R^{(2)}[i,1] \\ \vdots \\ R^{(2)}[i,n]\end{bmatrix},
	\end{equation*}
	where,
	\begin{equation*}
		R^{(2)} = \Softmax\left((W_K A^{(1)})^\top (W_Q A^{(1)})\right).
	\end{equation*}
	Observe that
	\begin{equation*}
		\left[(W_K A^{(1)})^\top (W_Q A^{(1)})\right]_{ij} = \langle W_Q \phi^{(1)}(r_j), W_K \phi^{(1)}(r_i) \rangle.
	\end{equation*}
	By Theorem \ref{theorem:function_class_inner_product_relnn}, composing an arbitrary learnable function \(\phi\) with inner products enables learning arbitrary relation functions on the input space (i.e.: any continuous, symmetric, positive semi-definite bivariate function. Here, the class of functions is actually larger since it allows for non-symmetric relation functions when \(W_Q \neq W_K\)).

	In the above, the space over which we are computing relations is itself a space of relation vectors. That is, \(\langle W_Q \phi^{(1)}(R_j), W_K \phi^{(1)}(R_i) \rangle\) computes a relation between object \(i\)'s relations and object \(j\)'s relations. Hence, choosing \(s_i^{(2)} = e_i\) and ignoring the Softmax for now, yields
	\begin{equation*}
		a_i^{(2)} = \phi^{(2)}\left(\begin{bmatrix}\langle W_Q \phi^{(1)}(r_i), W_K \phi^{(1)}(r_1) \rangle \\ \vdots \\ \langle W_Q \phi^{(1)}(r_i), W_K \phi^{(1)}(r_\m) \rangle\end{bmatrix}\right).
	\end{equation*}
	Thus, \(a_i^{(2)}\) computes arbitrary second-order relation functions. Namely, it computes arbitrary relations between object \(i\)'s relation vector and every other object's relation vector.

	More generally, at layer \(l\), we have
	\begin{align*}
		R^{(l)} &= \Softmax\left((W_K A^{(l-1)})^\top (W_Q A^{(l-1)})\right),\\
		a_i^{(l)} &= \phi^{(l)}\left(S^{(l)} \begin{bmatrix}R^{(l)}[i,1] \\ \vdots \\ R^{(l)}[i,n]\end{bmatrix}\right).
	\end{align*}
	Thus, \(R^{(l)}\) computes \(l\)th order-relations, and \(a_i^{(l)}\) is a linear map applied to the \(l\)th-order relations involving object \(i\).
\end{proof}


\subsection{Robustness and error correction}

For the relational cross-attention mechanism used by abstrators, an \(m\times m\) relation
is computed as  \(R = \mbox{Softmax}(K^T Q)\)
and relational cross attention then transforms the symbols by
\(A = SR\) so that each abstract variable \(a_j\) is in the convex hull of the set of symbols.
As long as \(S\) has rank \(m\), relations are uniquely determined from the abstract symbols.
Here we point out how the transformed symbols can be robust to noise if the symbols are
sufficiently redundant.

Specifically, suppose that the symbols \(S\) are transformed to \(A\) and corrupted with additive noise:
\begin{equation}
  A = SR + \Xi
\end{equation}
where a fraction \(\epsilon\) of the entries of \(\Xi\) are drawn from an adversarial noise distribution, and the other entries are zero; dropout noise is also possible.
This can be studied as an instance of compressed sensing and ``model repair'' \citep{candes_randall,model_repair}.  In particular, the relations can be recovered using the  robust regression estimator
\begin{equation}
  \hat r_j = \argmin_{u \in\reals^m} \| a_j - S u\|_1 \label{eq:lp}
\end{equation}
where \(A = (a_1,a_2,\ldots, a_m)\) with columns \(a_j\in\reals^d\).
The main lemma in \cite{model_repair} states that the following two conditions suffice:

\underline{Condition A:}
  There exists some \(\sigma^2\), such that for any fixed \(c_1,...,c_d\) satisfying \(\max_i|c_i|\leq 1\),
  \begin{equation}
    \left\|\frac{1}{d}\sum_{i=1}^d c_i s_{i\rdot} \right\|^2\leq \frac{\sigma^2 m}{d},
  \end{equation}
with high probability, where \(s_{i\rdot}\in\reals^m\) is the \(i\)th row of \(S\).

\underline{Condition B:}
  There exist \(\underline{\kappa}\) and \(\overline{\kappa}\), such that
  \begin{eqnarray}
  \label{eq:l1-upper-A} \inf_{\|\Delta\|=1}\frac{1}{d}\sum_{i=1}^d|s_{i\rdot}^T\Delta| &\geq& \underline{\kappa}, \\
  \label{eq:l2-upper-A} \sup_{\|\Delta\|=1}\frac{1}{d}\sum_{i=1}^d|s_{i\rdot}^T\Delta|^2 &\leq& \overline{\kappa}^2,
  \end{eqnarray}
  with high probability.

\begin{thm}\label{theorem:main-improved}
  Assume the symbol matrix \(S\) satisfies Condition A and Condition B. Then if
  \begin{equation}
  \frac{\overline{\kappa}\sqrt{\frac{m}{d}\log\left(\frac{e d}{k}\right)}+\epsilon\sigma\sqrt{\frac{m}{d}}}{\underline{\kappa}(1-\epsilon)}
  \end{equation}
  is sufficiently small, the linear program \eqref{eq:lp} recovers \(R\), so that \(\hat r_j = r_j\) with high probability.
  \end{thm}

The condition is essentially that
  \begin{equation}
    \frac{1}{1-\epsilon} \sqrt{\frac{m}{d}}
  \end{equation}
  is small, meaning that the dimension \(d\) of the symbols needs to be sufficiently large relative
  to the dimension \(k\) of the relation.

  \subsection{Sparse, high-dimensional relations}

 The above setting ensures enough redundancy to recover the relations, constraining the number of symbols \(k\) to be small relative to the symbol dimension \(d\). This is not appropriate in the situation where the relations are over a large number \(m\) of elements, for example, the contents of the entire episodic memory.
 In this setting we assume that the relation tensor \(R \in \reals^{m\times m}\) is sparse; that is,
 each column \(r_j \in \Delta_m\) has at most \(k\) nonzero entries: \(\|r_j\|_0 \leq k\). To recover the relation
 we now use the robust lasso estimator, which is a related linear program
\begin{equation}
  \hat r_j = \argmin_{u \in\reals^m} \| a_j - S u\|_1 + \lambda\|u\|_1. \label{eq:rlasso}
\end{equation}
Here we have an analogous theorem, stating that if
\begin{eqnarray}
  \frac{\overline{\kappa}/\underline{\kappa}}{1-\epsilon}\sqrt{\frac{k}{d}\log(2m)}\leq c,
\end{eqnarray}
for some sufficiently small constant \(c>0\), the robust lasso estimator \eqref{eq:rlasso} satisfies
\begin{equation}
  \|\hat r_j - r_j\| \leq C \frac{\overline{\kappa}/\underline{\kappa}^2}{1-\epsilon} \sqrt{\frac{\sigma^2 k}{d} \log(2m)}
\end{equation}
for some constant \(C\).
This implies that we can accurately recover the relation tensor in the high dimensional setting, even when many of the entries of the transformed abstract symbols are corrupted.


The above discussion shows how the relation tensor can be recovered from the transformed symbols, even under adversarial noise, assuming there is sufficient redundancy in the symbols. This implies that it is possible to predict as well from the transformed symbols as from the relations, without explicitly recovering the relations.
Using ideas from \citep{surfing,HandV17}, it may be possible to extend this theory to nonlinear mappings
\(y = \varphi(Au) + \eta\) where \(\varphi(\cdot)\) is an activation function.
%; this could serve as an alternative to the Hopfield model of memory retrieval and pattern completion.






\section{Experiments}\label{sec:experiments}
% \footnotetext{We ran the experiments described here on RTX 2080ti, RTX 3090, and A100 GPUs, available to us through our institution's internal cluster.The models here are relatively small; powerful GPUs are not required to train a single model. We found the use of GPUs useful for evaluating learning curves over several trials.}

\subsection{Discriminative relational tasks}\label{ssec:experiments_discriminative}

\subsubsection{Order relations: modeling asymmetric relations}\label{sssec:exp_order_relation}
A recent closely related work is~\citep{kerg2022neural}, which identifies and argues for certain inductive biases in relational models. They propose a model called CoRelNet which is, in some sense, the simplest possible model which satisfies those inductive biases. They show that this model is capable of outperforming several previous explicitly relation architectures. One inductive bias which they argue for is to model relations between objects as \textit{symmetric} inner products between object representations. %In this section, we aim to add to the discussion on inductive biases for relational learning by arguing that a general relational architecture needs to be able to model asymmetric relations as well.

The CoRelNet architecture is as follows: given a sequence of objects $(x_1, \ldots, x_m)$, embed them by some embedder $\phi$, then compute the similarity matrix $R = \text{Softmax}(A), A = {\left[\langle\phi(x_i), \phi(x_j)\rangle\right]}_{ij}$. In this form, CoRelNet models relations as necessarily symmetric. It can be modified to support asymmetric relations via the natural modification $A = {\left[\langle W_1 \phi(x_i), W_2 \phi(x_j)\rangle\right]}_{ij}$, where $W_1, W_2$ are trainable matrices. In~\citep{kerg2022neural} the authors argue that symmetry is an important inductive bias. The following simple experiment demonstrates that, while symmetry may be a useful inductive bias on relational tasks where the underlying relation is symmetric, the ability to model asymmetric relations is necessary for more general tasks.

% Another limitation is that it can only model single-dimensional relations---for each pair of objects $(i,j)$, their modeled relation is a single-dimensional scalar $R_{ij}$. The Abstractor is able to model a significantly larger class of relations. In particular, it is able to model asymmetric and multi-dimensional relations through the $\text{MultiHeadRelation}$ operation. This is demonstrated by the following simple experiment.

We generate $N = 32$ ``random objects'' represented by iid Gaussian vectors, $o_i \overset{iid}{\sim} \mathcal{N}(0,
I_d) \in \mathbb{R}^d$, and associate an order relation to them $o_1 \prec o_2 \prec \cdots \prec o_N$. Note that $\prec$ is \textit{not symmetric}. Of the $N^2 = 1024$ possible pairs $(o_i, o_j)$, 15\% are held out as a validation set (for early stopping) and 35\% as a test set. We evaluate learning curves by training on the remaining 50\% and computing accuracy on the test set (10 trials for each training set size). Note that under this set up, we are evaluating the models on pairs they have never seen. Thus, the models will need to generalize based on the transitivity of the $\prec$ relation.

We compare three models: an Abstractor, standard (symmetric) CoRelNet, and asymmetric CoRelNet. We observe that standard symmetric CoRelNet is completely unable to learn the task, whereas the Abstractor and asymmetric CoRelNet learn the transitive $\prec$ relation~(\Cref{fig:exp_order_relation}).~\citep{kerg2022neural} finds symmetry to be an important inductive bias for the relational tasks they consider (which are based on a simple same/different underlying relation), whereas we find that the ability to model asymmetric relations is necessary for more general tasks. This can be explained in terms of the relational bottleneck. In a relational task, there exists a sufficient statistic which is relational. The relational bottleneck restricts the space of learnable representations to be the space of relational features, thus making the search for a good representation easier and more sample efficient. In a relational task in which there exists a sufficient statistic involving a symmetric relation (as is the case with same/different tasks), we can further restrict the search space to be over symmetric relational representations, which should make learning a good representation even more sample efficient. Since the experiments \citep{kerg2022neural} considers are indeed symmetric, symmetry was a useful inductive bias. However, some relations are more complex and may be asymmetric---an example is order relations. Symmetric inner products don't have the representational capacity to model such relations. But asymmetric inner products with different learned left and right encoders can model such relations.

\subsubsection{SET: modeling multi-dimensional relations}\label{sssec:exp_set}

The $\prec$ relation is a one-dimensional relation. Similarly, the underlying relations in the experiments considered in~\citep{kerg2022neural} are also one-dimensional (same/different). In the next experiment, we explore a discriminative relational tasks which relies on a multi-dimensional relation. The task we consider is SET. 

\begin{wrapfigure}{R}{0.25\textwidth}
	\vskip-5pt
	\begin{tabular}{c}
		\includegraphics[width=.25\textwidth]{figures/set_example}\\[-5pt]
	\end{tabular}
	\caption{\footnotesize The SET game}
\end{wrapfigure}

SET is a relatively straightforward but challenging cognitive task that engages reasoning faculties in a deliberative, attentionally directed manner, requiring several levels of abstraction over sensory information. Players are presented with a sequence of cards, each of which contains figures that vary along four dimensions (color, number, pattern, and shape) and they must find triplets of cards which obey a deceptively simple rule: along each dimension, cards in a SET must either have the same value or three unique values. For example, in the figure to the right, the cards with two solid blue/purple diamonds, two striped blue squiggles, and two open blue oblongs form a SET: same color, same number, different patterns, different shapes.

In this experiment, the task is, given a triplet of card images, to determine whether they form a SET or not. A CNN classifier is trained on the card images to predict the four attributes. An intermediate layer is used as an embedder for all relational models. We compare an Abstractor model to CorelNet. The shared architecture is $\texttt{CNN} \to \{\texttt{Abstractor} \text{ or } \texttt{CorelNet}\} \to \texttt{Flatten} \to \texttt{Dense}$. We report learning curves in~\Cref{fig:exp_set_classification} (10 trials per training set size). We find that the Abstractor model significantly out-performs CoRelNet. We attribute this to its ability to model multi-dimensional relations. In this task, there exists four different relations (one for each attribute) which are needed to determine whether a triple of cards forms a set. CoRelNet would need to squeeze all of this information into a single-dimensional scaler whereas the Abstractor can model each relation separately. We hypothesize that the ability to model relations as multi-dimensional is also the reason that the Abstractor can learn the order relation better than asymmetric CoRelNet---even though the underlying relation is one-dimensional, having a multi-dimensional representation enables greater robustness and multiple avenues towards a good solution during optimization.

\begin{figure}[t]
    % \vskip-.2in
    \centering
    \begin{subfigure}[t]{0.45\textwidth}
        \centering
        %\hskip-.35in
        \includegraphics[width=\textwidth]{figures/experiments/pairwise_order_learning_curves.pdf}
        % \vskip-5pt
        \caption{The $\prec$ relation can be learned with asymmetric but not symmetric inner products.}\label{fig:exp_order_relation}
    \end{subfigure} 
    \hskip10pt
    % \captionsetup[subfigure]{oneside,margin={-.3in,0in}}
    \begin{subfigure}[t]{0.45\textwidth}
        \centering
        % \vskip10pt
        % \hskip-.6in
        \includegraphics[width=\textwidth]{figures/experiments/set_classification.pdf}
        % \vskip-5pt
        \caption{The Abstractor's ability to model multi-dimensional relations enables it to solve SET.}\label{fig:exp_set_classification}
    \end{subfigure}
    \caption{Experiments on discriminative relational tasks and comparison to CoRelNet.}
\end{figure}


\subsection{Object-sorting: (fully) relational sequence-to-sequence tasks}\label{ssec:experiments_object_sorting}

In the following set of experiments, we consider sequence-to-sequence relational tasks, and compare an Abstractor-supported model to a plain Transformer. We consider the task of \textit{sorting} sequences of randomly permuted random objects. These experiments are ``fully relational'' in the sense that there exists a relation (order) which is a sufficient statistic for solving the task---the features of individual objects beyond this relation are extraneous. This is a more controlled setting which validates the hypothesis that the inductive biases of the Abstractor indeed confer benefits in modeling relations. We consider the more ``realistic'' setting of partially-relational tasks in the next section. The experiments in the present section demonstrate that the Abstractor enables a dramatic improvement in sample-efficiency on sequence-to-sequence relational tasks.


\begin{figure}[ht]
    \begin{subfigure}[t]{0.40\textwidth}
        %\centering
        \hskip-.35in\includegraphics[scale=.95]{figures/experiments/random_object_sorting.pdf}
        \vskip-5pt
        \caption{Learning curves on sorting sequences of random objects. The abstractor is dramatically more sample-efficient.}\label{fig:exp_object_sorting}
    \end{subfigure}\hspace{\fill}
    \begin{subfigure}[t]{0.40\textwidth}
        %\centering
        \hskip-.6in\includegraphics[scale=.95]{figures/experiments/random_object_sorting_generalization.pdf}
        \vskip-5pt
        \caption{Learning curves with and without pre-training on a similar sorting task. The Abstractor benefits significantly from pre-training.}\label{fig:exp_object_sorting_generalization}
    \end{subfigure}
    % \bigskip

    \begin{subfigure}[t]{0.40\textwidth}
        %\centering
        \hskip-.35in\includegraphics[scale=.95]{figures/experiments/additive_robustness.pdf}
        \vskip-5pt
        \caption{The Abstractor is more robust to corruption by additive noise. }\label{fig:exp_robustness}
    \end{subfigure}\hspace{\fill}
    \begin{subfigure}[t]{0.40\textwidth}
        %\centering
        \hskip-.6in\includegraphics[scale=.95]{figures/experiments/multiplicative_robustness.pdf}
        \vskip-5pt
        \caption{The Abstractor is more robust to corruption by a random linear transformation.}\label{fig:exp_robustness2}
    \end{subfigure}

    \caption{Experiments on fully relational sequence-to-sequence tasks.}\label{fig:experiments}
    \vskip-.10in
\end{figure}

\subsubsection{Superior sample-efficiency on relational seq2seq tasks compared to standard transformers}

We generate random objects in the following way. First, we generate two sets of random attributes $\mathcal{A} = \{a_1, a_2, a_3, a_4\}$, $a_i \overset{iid}{\sim} \mathcal{N}(0, I) \in \mathbb{R}^{4}$ and $\mathcal{B} = \{b_1, \ldots, b_{12}\}$, $b_i \overset{iid}{\sim} \mathcal{N}(0, I) \in \mathbb{R}^{8}$. To each set of attributes, we associate the strict ordering relation $a_1 \prec a_2 \prec a_3 \prec a_4$ and $b_1 \prec b_2 \prec \cdots \prec b_{12}$, respectively. Our random objects are formed by the Cartesian product of these two attributes $\mathcal{O} = \mathcal{A} \times \mathcal{B}$, yielding $N = 4 \times 12 = 48$ objects (i.e.: each object in $\mathcal{O}$ is a vector in $\mathbb{R}^{4+8}$ formed by a concatenation of one attribute value in $\mathcal{A}$ and one in $\mathcal{B}$). Then, we associate with $\mathcal{O}$ the strict ordering relation corresponding to the order relation of $\mathcal{A}$ as the primary key and the order relation of $\mathcal{B}$ as the secondary key. i.e.: $(a_i, b_j) \prec (a_k, b_l)$ if $a_i \prec a_k$ or if $a_i = a_k$ and $b_j \prec b_l$.

Given a set of objects in $\mathcal{O}$, the task is to sort it according to $\prec$. More precisely, the input sequences are randomly permuted sequences of $10$ objects in $\mathcal{O}$ and the target sequences are the indices of the object sequences in sorted order (i.e., the `argsort'). The training data are sampled uniformly from the set of length-10 sequences in $\mathcal{O}$. We also generate a non-overlapping validation dataset (used during training for early stopping) and a testing dataset (used during evaluation).

We evaluate learning curves on an Abstractor, a Transformer, and an ``Ablation'' model (10 trials for each training set size). The Abstractor uses the architecture $\texttt{Encoder} \to \texttt{Abstractor} \to \texttt{Decoder}$. The Encoder-to-Abstractor interface uses relational cross-attention and the Abstractor-to-Decoder interface uses standard cross-attention. The Ablation Model aims to test the effects of the relational cross-attention in the Abstractor model---it is architecturally identical to the Abstractor model with the crucial exception that the Encoder-to-Abstractor interface instead uses standard cross-attention. The hyperparameters of the models are chosen so that the parameter counts are similar. % TODO: add details here? or in supplement?
We find that the Abstractor is dramatically more sample-efficient than the Transformer and the Ablation model~(\Cref{fig:exp_object_sorting}).

\subsubsection{Ability to generalize to similar tasks}

Continuing with the object-sorting task and the dataset generated as described above, we test the Abstractor's ability to generalize from similar relational tasks through pre-training. The main task uses the same dataset described above. The pre-training task involves the same object set $\mathcal{O}$ but the order relation is changed. The ordering in attribute $\mathcal{A}$ is randomly permuted, while the ordering in attribute $\mathcal{B}$ is kept the same. A strict ordering relation $\prec$ on $\mathcal{O}$ is obtained in the same way---using the order in $\mathcal{A}$ as the primary key and the order in $\mathcal{B}$ as the secondary key.

The Abstractor model here uses the architecture $\texttt{Abstractor} \to \texttt{Decoder}$ (i.e.: no Transformer
encoder), and the transformer is the same as the previous section. We pre-train both models on the pre-training task
and then, using those learned weights for initialization, evaluate learning curves on the original task. Since the
Transformer requires more training samples to learn the object-sorting task, we use a pre-training set size of $3,000$, chosen based on the results of the previous subsection so that it is large enough for the Transformer to learn the pre-training task. This experiment assesses the models' ability to generalize relations learned on one task to a new task.~\Cref{fig:exp_object_sorting_generalization} shows the learning curves for each model with and without pre-training. We observe that when the Abstractor is pre-trained, its learning curve on the object-sorting task is significantly accelerated. The transformer does not benefit from pre-training.

\subsubsection{Robustness and Out-of-Distribution generalization}
In this experiment, we evaluate robustness to a particular type of noisy corruption. We train each model on the same
object-sorting task described above. We use a fixed training set size of $3,000$ for the same reason
---it is large enough that all models are able to learn the task. On the hold out test set, we corrupt the object
representations by applying a random linear transformation. In particular, we randomly sample a random matrix the
entries of which are iid zero-mean Gaussian with variance $\sigma^2$, $\Phi \in \mathbb{R}^{d \times d}, \Phi_{ij} \sim \mathcal{N}(0, \sigma^2)$. Each object in $\mathcal{O}$ is then corrupted by this random linear transformation,
$\tilde{o}_i = \Phi o_i, \ \text{ for each } i \in [48]$. We also test robustness to additive noise via $\tilde{o}_i = o_i + \varepsilon_i, \varepsilon_i \sim \mathcal{N}(0, \sigma^2 I_d)$.

The models are evaluated on the hold-out test set with objects replaced by their corrupted version. We evaluate the sorting accuracy of each model while varying the noise level $\sigma$ (5 trials at each noise level). The results are shown in figures~\ref{fig:exp_robustness} and~\ref{fig:exp_robustness2}. We emphasize that the models are trained only on the original objects in $\mathcal{O}$, and are not trained on objects corrupted by any kind of noise.

This experiment can be interpreted in two lights: the first is robustness to noise. The second is a form of out-of
-distribution generalization. Note that the objects seen by the models post-corruption lie in a different space than
those seen during training. Hence the models need to learn relations that
are in some sense independent of the value representation.
As a theoretical justification for this behavior,~\cite{zhouCompressedPrivacySensitive2009} shows that $\langle \Phi x, \Phi y \rangle \approx \langle x, y \rangle$ in high dimensions, for a random matrix $\Phi$ with iid Gaussian entries. This indicates that models whose primary computations are performed via inner products, like Abstractors, may be more robust to this kind of corruption.


\subsection{Math problem-solving: partially-relational sequence-to-sequence tasks}\label{ssec:experiments_math}

The object-sorting experiments in the previous section are ``purely relational'' in the sense that the set of pairwise $\prec$ relations is a sufficient statistic for solving the task (i.e., the target sequence is conditionally independent of the input sequence given the relations). In general, in a generative sequence-to-sequence task, there may not be a relation which is a sufficient statistic. Nonetheless, relational reasoning may still be crucial for solving the task, and the enhanced relational reasoning capabilities of the Abstractor may enable performance improvements. The ``partially-relational'' architectures described in~\Cref{sec:abstractor_architectures} enable a branch of the model to focus on relational reasoning while maintaining a connection to object-level attributes. In this section, we compare an Abstractor model using architecture (d) of~\Cref{fig:abstractor_architectures} to standard Transformers on a set of math problem-solving tasks based on the dataset proposed in~\citep{saxtonAnalyzingMathematicalReasoning2019}.

\begin{figure}
    \begin{center}
    \begin{small}
    \begin{tabular}{cc}
        \begin{tabular}{l}
        Task: \texttt{polynomials\_\_expand}\\
        Question: \texttt{Expand (2*x + 3)*(x - 1).}\\
        Answer: \texttt{2*x**2 + x - 3}
        \end{tabular}
        &
        \begin{tabular}{l}
        Task: \texttt{algebra\_\_linear\_1d}\\
        Question: \texttt{Solve for z: 5*z + 2 = 9.}\\
        Answer: \texttt{7/5}
        \end{tabular}
    \end{tabular}
    \end{small}
    \end{center}
\caption{Examples of input/target sequences from the math problem-solving dataset.}\label{fig:math_dataset}
\end{figure}

The dataset consists of several math problem-solving tasks, with each task having a dataset of question-answer pairs. The tasks include solving equations, expanding products of polynomials, simplifying polynomials, differentiating functions, predicting the next term in a sequence, etc. A sample of question-answer pairs is displayed in~\Cref{fig:math_dataset}. The overall dataset contains $2 \times 10^6$ training examples and $10^4$ validation examples per module. Questions have a maximum length of 160 characters and answers have a maximum length of 30 characters. We use character-level encoding with a common alphabet of size $95$ (including upper/lower case characters, digits, and punctuation). Each question and answer is tokenized and padded with the null character.

We compare an Abstractor model using architecture (d) in~\Cref{fig:abstractor_architectures} to a standard Transformer. Since the Abstractor-based model with architecture (d) has an Abstractor module in addition to an Encoder and Decoder, we compare against two versions of the Transformer in order to control for parameter count. In the first, the Encoder/Decoder have identical hyperparamters to the Abstractor model. In the second, we increase the model dimension and hidden layer size of the feedforward network such that the overall parameter count is approximately the same as the Abstractor model. We refer to the first model as ``Transformer'' and the second as ``Transformer+'' in the figures. We train on modestly sized models. The precise architectural details and hyperparamters are in~\Cref{tab:math_architecture}. The text is 

\begin{wrapfigure}{R}{0.5\textwidth}
    \centering
    \includegraphics[width=0.5\textwidth]{figures/experiments/math_metrics.pdf}
    \caption{\footnotesize End-of-training teacher-forcing accuracy.}\label{fig:math_metrics}
\end{wrapfigure}

We evaluate the three models on 5 tasks within the dataset, differentiating functions (calculus), predicting the next term in a sequence (algebra), solving a linear equation (algebra), expanding polynomials, and adding polynomials. For each, we train on the training split and track the teacher-forcing accuracy (excluding null characters) on the validation split. We train each model for 50 epochs with the categorical crossentropy loss and the Adam optimizer using a learning rate of $6 \times 10^{-4}$, $\beta_1 = 0.9, \beta_2 = 0.995, \varepsilon = 10^{-9}$. For each pair of model and task, we repeat the experiment 5 times and compute error bars as twice the standard error of mean.

\Cref{fig:math_metrics} shows the end-of-training teacher-forcing accuracy on the validation split for each task.~\Cref{fig:math_training_curves} shows the validation teacher-forcing accuracy during the course of training. We observe a modest improvement in accuracy compared to both `Transformer' and `Transformer+' across all tasks. The larger Transformer tends to perform better than the smaller Transformer, but the Abstractor-based model consistently outperforms both. This indicates that the performance improvement stems from the architectural modification rather than merely the increase in parameter count. We conjecture that a ``partially-relational'' Abstractor architecture (e.g., architecture (d)) implements two branches of information-processing. The Encoder performs more general-purpose processing of the input sequence, while the Abstractor performs more specialized relational processing. The Decoder then has access to both representations, enabling it to perform the task more effectively.

\begin{figure}[t]
    \centering
    \includegraphics[width=\textwidth]{figures/experiments/math_training_curves.pdf}
    \caption{Training curves comparing an Abstractor-based architecture to a standard Transformer on mathematics problem-solving tasks.}\label{fig:math_training_curves}
\end{figure}


% \subsection{Modularity and comparison to purely symbolic representations}
\label{ssec:set_exp}


\begin{wrapfigure}{R}{0.25\textwidth}
	\vskip-5pt
	\begin{tabular}{c}
		\includegraphics[width=.25\textwidth]{figures/set_example}\\[-5pt]
	\end{tabular}
	\caption{\footnotesize The SET game}
\end{wrapfigure}
SET is a relatively straightforward but challenging cognitive task that engages reasoning faculties in a deliberative
, attentionally directed manner, requiring several levels of abstraction over sensory embeddings. Players are
presented with 12 cards, each of which contains figures that vary along four dimensions (color, number, pattern, and
shape) and they must find subsets of three cards which obey a deceptively simple rule: along each dimension, all cards in a SET must either have the same or unique values.
% JDC: DO FIGURES NEED TO BE NUMBERED?
For example, in the figure to the right, the cards with two solid blue/purple diamonds, two striped blue squiggles, and two open blue oblongs form a SET: same color, same number, different patterns, different shapes.

To simulate the task of deciding if a triple forms a SET, we first train a convolutional neural network to process the color images of the cards (a full deck includes 81 cards). The CNN is trained to predict the attribute of
each card, as a multi-label classification, and then an embedding of dimension $d=32$ of 
each card is obtained. This embedding layer uses an \MLP{} to map the convolutional feature maps into a distributed
representation. Next, we train Abstractors separately for each of the four attributes to learn same/different
relations, where the task is to decide if an input pair of cards is the same or different for that attribute. 
We then use the query and key mappings $W_Q$ and $W_K$ learned for these relations to initialize the relations
in a multi-head Abstractor. The Abstractor is then trained on a dataset of triples of cards, half of which form a SET. 

This is compared to a baseline symbolic model where, instead of images, the input is a vector with 12 bits,
explicitly encoding the relations. That is, for each of the four attributes, a binary symbol is computed for each pair of three input cards---1 if the pair is the same in that attribute, and 0 otherwise. A two-layer MLP is then trained to decide if the triple forms a SET. The MLP using the symbolic representation can be considered as a lower bound on the performance achievable by the Abstractor. This comparison shows that the Abstractor is able to solve a task using relations learned in other tasks (modularity), with a sample efficiency that is not far from that obtained
with purely symbolic, noise-free encodings of the relevant relations. We note that this uses a simple Abstractor configuration, without self-attention.

% JDC: I REALIZE WE ARE TIGHT ON SPACE, BUT IF THERE IS ROOM, COULD ADD THE FOLLOWING, OR A SHORTENED FORM OF
This subtask suggets how Abstractors might be viewed as an intermediate between strong ``nativist'' approaches that assume a pre-existing foundation of symbolic primitives and purely ``empiricist'' approaches that assume
similar capabilities can emerge simply from processing large amounts of data \cite{howtogrowamind}.
%can be achieved simply by the application of general purpose learning algorithms to large
%amounts of data -- showing how the latter, agumented with simple architectural inductive biases and trained using
%plausible and practical forms of curricular learninng both to generate genuinely symbolic representations, and use these to achieve the flexibilty and efficiency characteristic of human reasoning.

 % incorporated this into section 4.1.2 for now.
\section{Summary}\label{sec:discuss}

In this work we have proposed a framework which extends Transformer models to naturally support types of relational learning, through a cross-attention mechanism that enforces a relational bottleneck, separating relational information from object-level attributes. Building on insights gained from the implementation of a relational bottleneck in other forms \citep{esbn, kerg2022neural}, this exploits the powerful attentional capabilities of the transformer architecture to identify relevant relationships. Our experiments validate that the proposed architecture enables dramatic improvements in relational processing, and that this has the potential to translate into meaningful gains in more general sequence modeling tasks.
% Experiments with controlled purely relational tasks as well as more general sequence modeling tasks indicate that this framework has the potential to combine the benefits of function approximation over sensory states, as exploited in many deep learning models, with abstraction and relational reasoning abilities supported by symbolic processing.
Interesting work remains to better understand the potential of this framework, and how it may relate to the algorithms of human cognition as implemented in the brain.

\subsection*{Code and Data}
Code, experimental logs, and instructions for reproducing our results are available at: \url{https://github.com/Awni00/abstractor}

\medskip

% \clearpage
{%\small
\setlength{\bibsep}{8pt plus 0.3ex}
\bibliographystyle{apalike}
\bibliography{references}
}


\clearpage
\appendix
\section{Multi-Attention Decoder}\label{sec:multi_attn_decoder}

\begin{algorithm}[ht!]
	\caption{Multi-Attention Decoder}\label{alg:multi_attention_decoder}
	\SetKwInOut{Input}{Input}
	% \SetKwInOut{Output}{Output}
	% \SetKwInOut{LearnableParams}{Learnable parameters}
	% \SetKwInOut{HyperParams}{Hyperparameters}

	\Input{
        Target sequence: $\bm{y} = (y_0, \ldots, y_{l_y-1})$, \\
        Context sequences: $X^{(i)} = (x_1^{(i)}, \ldots, x_{l_i}^{(i)}), \ i=1, \ldots, K$
        }
	\vspace{1em}

    $D^{(0)} \gets \bm{y}$

	\For{\(l \gets 1\) \KwTo \(L\)}{

        $D^{(l)} \gets \mathrm{CausalSelfAttention}\paren{D^{(l-1)}}$

        \texttt{residual connection and layer-normalization}


        \For{\(i \gets 1\) \KwTo \(K\)}{
            $D^{(l)} \gets \mathrm{CrossAttention}\paren{D^{(l)}, X^{(i)}}$

            \texttt{residual connection and layer-normalization}
        }

        $D^{(l)} \gets \mathrm{FeedForward}\paren{D^{(l)}}$

        }

    \textbf{Output:} $D^{(L)}$

\end{algorithm}

\section{Experimental details}\label{sec:experimental_details}

\subsection{Discriminative tasks (\Cref{ssec:discriminative_tasks})}

\textbf{Abstractor architecture}

\textbf{CoRelNet Architecture}



\subsection{SET}

\textbf{Common embedder's architecture}

\textbf{Abstractor architecture}

\textbf{CoRelNet Architecture}
\section{Additional Experiments}

\subsection{Object-sorting: Robustness and Out-of-Distribution generalization}
In this experiment, we evaluate robustness to a particular type of noisy corruption. We train each model on the same
object-sorting task described above. We use a fixed training set size of $3,000$ for the same reason---it
is large enough that all models are able to learn the task. On the hold out test set, we corrupt the object
representations by applying a random linear transformation. In particular, we randomly sample a random matrix the
entries of which are iid zero-mean Gaussian with variance $\sigma^2$, $\Phi \in \mathbb{R}^{d \times d}, \Phi_{ij} \sim \mathcal{N}(0, \sigma^2)$. Each object in $\mathcal{O}$ is then corrupted by this random linear transformation,
$\tilde{o}_i = \Phi o_i, \ \text{ for each } i \in [48]$. We also test robustness to additive noise via $\tilde{o}_i = o_i + \varepsilon_i, \varepsilon_i \sim \mathcal{N}(0, \sigma^2 I_d)$.

The models are evaluated on the hold-out test set with objects replaced by their corrupted version. We evaluate the sorting accuracy of each model while varying the noise level $\sigma$ (5 trials at each noise level). The results are shown in figures~\ref{fig:exp_robustness1} and~\ref{fig:exp_robustness2}. We emphasize that the models are trained only on the original objects in $\mathcal{O}$, and are not trained on objects corrupted by any kind of noise.

This experiment can be interpreted in two lights: the first is robustness to noise. The second is a form of out-of
-distribution generalization. Note that the objects seen by the models post-corruption lie in a different space than
those seen during training. Hence the models need to learn relations that
are in some sense independent of the value representation.
As a theoretical justification for this behavior,~\cite{zhouCompressedPrivacySensitive2009} shows that $\langle \Phi x, \Phi y \rangle \approx \langle x, y \rangle$ in high dimensions, for a random matrix $\Phi$ with iid Gaussian entries. This indicates that models whose primary computations are performed via inner products, like Abstractors, may be more robust to this kind of corruption.

\begin{figure}[ht]
    \begin{subfigure}[t]{0.40\textwidth}
        %\centering
        \hskip-.35in\includegraphics[scale=.95]{figures/experiments/additive_robustness.pdf}
        \vskip-5pt
        \caption{The Abstractor is more robust to corruption by additive noise. }\label{fig:exp_robustness1}
    \end{subfigure}\hspace{\fill}
    \begin{subfigure}[t]{0.40\textwidth}
        %\centering
        \hskip-.6in\includegraphics[scale=.95]{figures/experiments/multiplicative_robustness.pdf}
        \vskip-5pt
        \caption{The Abstractor is more robust to corruption by a random linear transformation.}\label{fig:exp_robustness2}
    \end{subfigure}
    \caption{Experiments on robustness.}\label{fig:exp_robustness}
\end{figure}
% 
%\newcommand{\MLP}{\text{MLP}}
%\newcommand{\FeedForward}{\text{FeedForward}}
%\newcommand{\Softmax}{\text{Softmax}}
%\newcommand{\reals}{\mathbb{R}}
\def\m{m}


\section{Relational symbolic message-passing}
\label{sec:message_passing}

At a high level, the primary function of a relational abstractor is to compute abstract relational features of its inputs. That is, given a set of input objects $o_1, \ldots, o_\m$, the relational abstractor computes a function on the set of pairwise relations between objects $\{ R(o_i, o_j) \}_{i,j}$, where $R(\cdot, \cdot)$ is a relation between a pair of objects; the relations are learned to carry out a specific prediction task.

At the core of relational abstractors is an operation we refer to as \textit{relational symbolic message-passing}. The input to this operation is a relation tensor $R = \left[R(o_i, o_j)\right]_{i,j=1}^\m$, where $R(o_i, o_j) \in \mathbb{R}^{d_r}$ is a vector describing the relation between object $o_i$ and object $o_j$. We will come back to how an abstractor computes the relation tensor.

The first set of learnable parameters of symbolic message-passing is a set of symbols $s_1, \ldots, s_\m \in \mathbb{R}^{d_s}$, where the hyperparameter $d_s$ is the dimension of the symbolic vectors. We call these parameters \textit{symbols} because each of them references (or ``is bound to") a particular object, but they are independent of the values of these objects. That is, the $i$th symbol references the $i$th object, but the value of $s_i$ is independent of the value of $o_i$. The use of those learned input-indpendent symbols is how symbolic message-passing achieves its abstraction.

In relational symbolic message-passing, we perform message-passing on these learned symbolic parameters according to the relation tensor $R$. In general, this message-passing operation can be described as a set-valued function of the form
\begin{equation}
    \label{eq:symbolic_message_passing}
    s_i \leftarrow \text{Update}\Big( s_i, \ \left\{ \left(R[i,j], s_j\right)\right\}_{j\in[m]}\Big), \quad i = 1, \ldots, m
\end{equation}
That is, the value of the $i$th symbol is updated as a function of the set of tuples $(R[i,j], s_j)$ of the relations with all other objects and the symbols of these objects. The symbols $s_j$ are naturally viewed
as values on the nodes of a graph, and the relations $R[i,j]$ are naturally viewed as weights on the edges. A simple but important special case of this is
\begin{equation}
    \label{eq:linear_symbolic_mp}
    s_i \leftarrow \sum_{j=1}^{m} R[i,j] s_j, \quad i=1, \ldots, m
\end{equation}
In the above, suppose that $d_r = 1$. Otherwise, the above operation is done for each dimension of the relation $R$ and the result is concatenated, as in multi-head attention.

Following message-passing, each updated symbol $s_i$, can be passed through a feedforward neural network $f:\reals^{d_s}\rightarrow \reals^{d_s}$ to compute a non-linear function of the output. %Empirically, a residual connection and layer normalization may be useful, as in a transformer.
This message-passing operation can be repeated multiple times to iteratively update the symbolic vectors.  The output of relational symbolic message-passing is the set of symbols $A$ at the end of this sequence of message-passing operations. The overall procedure is summarized in Algorithm~\ref{alg:relational_abstractor}. In Section~\ref{sec:function_spaces} we characterize the class of functions on relations that this operation can compute.

\begin{algorithm}[th!]
    \caption{Relational Abstractor}\label{alg:relational_abstractor}
    \SetKwInOut{Input}{Input}
    \SetKwInOut{Output}{Output}
    \SetKwInOut{LearnableParams}{Learnable Parameters}
    \SetKwInOut{HyperParams}{Hyper Parameters}

    \Input{Encoder entities: $E = (z_1, \ldots, z_\m) \in \mathbb{R}^{d_e \times m}$}
    \HyperParams{$L$ (number of layers), $H$ (number of heads), hyperparameters of feedforward network}
    \LearnableParams{symbols $S \in \reals^{d_s \times m}$, parameters $\{\theta_l\}$ of attention head and feedforward models}
    \Output{Abstracted sequence: $A = (a_1, \ldots, a_\m) \in \reals^{d_a \times m}$}
    \vspace{1em}

    $A \gets S$

    \For{$l \gets 1$ \KwTo $L$}{
        $A \gets \text{SelfAttention}_{\theta_l}(A)$

        $A \gets \crossattend{E}{E}{A}{\theta_l}$

        $A \gets\FeedForward_{\theta_l}(A)$
        }
\end{algorithm}

\subsection{Computing the relation tensor with relational cross-attention}

Symbolic message passing is the first of the two main ingredients of abstractors. What remains is to describe how the abstractor computes the relation tensor $R$. This is done through a variant of transformer cross-attention that we refer to as \textit{relational cross-attention}.
To motivate this, we first describe how we can compute relations between pairs of objects through inner products. The inner product operation is a natural way to capture notions of relations and similarity. In Euclidean space, inner products capture the geometric alignment between vectors. Similarly, for objects with vector representations, inner products between these vector representations can capture relations between these objects.

In general, we can formulate inner product relations as the standard Euclidean inner product between a pair of transformed object vectors. That is, $R(o_i, o_j) = \langle \phi(o_i), \psi(o_j) \rangle$. This captures a large class of relation functions. In particular, the theory of reproducing kernel Hilbert spaces implies that any continuous, symmetric function on pairs of objects can be approximated with such functions
\citep{universal}. Multi-dimensional relation functions can be achieved by stacking multiple such inner products.

We will take the transformations $\phi$ and $\psi$ to be linear, suggestively denoting them $W^q$ and $W^k$, respectively. For the message-passing operation \Cref{eq:linear_symbolic_mp}, it is useful to normalize the relations $R$ via a softmax so that they are non-negative and sum to one. Hence, after each symbolic message-passing operation, the updated representation of each symbol involves a convex combination of the other symbols that is determined by the relations. This can be compactly written as
\begin{equation}
    \label{eq:relational_crossattention}
    \begin{split}
        A &= S R, \\
        R &= \Softmax\left((W^k E)^\top (W^q E)\right),
    \end{split}
\end{equation}
where $S = (s_1, \ldots, s_\m) \in \reals^{d_s \times m}$ is the matrix of symbols and $E
= (o_1, \ldots, o_\m) \in \reals^{d_o \times m}$ is the matrix of embeddings of input objects. This is essentially the cross-attention operation of transformers, where the queries and keys both come from the input objects, and the values come from the learned input-independent symbols. Hence, we refer to this operation as relational cross-attention and denote it by $\qkv{E}{E}{S}$. %\text{CrossAttention}(Q \gets E, K \gets E, V \gets $)$

Using relations and message passing as the key operations, the relational abstractor
framework is described in Algorithm~\ref{alg:relational_abstractor}. Our presentation suggests
a close connection to transformers; we further develop this connection in the following section.




% NOTE / TODO: need to describe what we mean by `relations', `relation functions', `inner product relations', etc. in more detail somewhere?
% maybe in intro at high-level

% comment this out
% \end{document}

% 
\section{Function classes}
\label{sec:function_spaces}

\def\rdot{\bigcdot}
\def\F{{\mathfrak{F}}}

In this section we discuss function classes for relational learning and symbolic message passing.

\subsection{Robustness and error correction}

For the relational cross-attention mechanism used by abstrators, an $m\times m$ relation 
is computed as 
\begin{equation}
  R = \mbox{Softmax}(K^T Q)
\end{equation}
where $Q = W_Q E\in \reals^{d\times m}$ and $K = W_K E\in \reals^{d\times m}$ are the query and keys; the softmax is applied column-wise. So, each column $r_j \in \Delta_k$ is in the simplex of 
non-negative $k$-vectors summing to one. Let $(s_1,\ldots, s_m) = S\in\reals^{d\times m}$ be
a matrix of symbols. The relational cross attention then transforms the symbols by 
\begin{equation}
  A = SR
\end{equation}
so that each abstract variable $a_j$ is in the convex hull of the set of symbols.
As long as $S$ has rank $m$, this a linear transformation and the relations are 
uniquely determined from the abstract symbols. 

More generally, suppose that the symbols $S$ are transformed to $A$ and corrupted with additive noise:
\begin{equation}
  A = SR + \Xi
\end{equation}
where a fraction $\epsilon$ of the entries of $\Xi$ are drawn from an adversarial noise distribution, and the other entries are zero; dropout noise is also possible. 

This can be studied as an  instance of compressed sensing and ``model repair'' \citep{model_repair}.  In particular, we can use the  robust regression estimator
\begin{equation}
  \hat r_j = \argmin_{u \in\reals^k} \| a_j - S u\|_1 \label{eq:lp}
\end{equation}
where $A = (a_1,a_2,\ldots, a_k)$ with columns $a_j\in\reals^d$.
The main lemma in the model repair paper states that the following two conditions suffice:

\underline{Condition A:}
  There exists some $\sigma^2$, such that for any fixed $c_1,...,c_d$ satisfying $\max_i|c_i|\leq 1$,
  \begin{equation}
    \left\|\frac{1}{d}\sum_{i=1}^d c_i s_{i\rdot} \right\|^2\leq \frac{\sigma^2 k}{d},
  \end{equation}
with high probability, where $s_{i\rdot}\in\reals^k$ is the $i$th row of $S$.
  
\underline{Condition B:}
  There exist $\underline{\kappa}$ and $\overline{\kappa}$, such that
  \begin{eqnarray}
  \label{eq:l1-upper-A} \inf_{\|\Delta\|=1}\frac{1}{d}\sum_{i=1}^d|s_{i\rdot}^T\Delta| &\geq& \underline{\kappa}, \\
  \label{eq:l2-upper-A} \sup_{\|\Delta\|=1}\frac{1}{d}\sum_{i=1}^d|s_{i\rdot}^T\Delta|^2 &\leq& \overline{\kappa}^2,
  \end{eqnarray}
  with high probability.

\begin{thm}\label{thm:main-improved}
  Assume the symbol matrix $S$ satisfies Condition A and Condition B. Then if
  \begin{equation}
  \frac{\overline{\kappa}\sqrt{\frac{k}{d}\log\left(\frac{e d}{k}\right)}+\epsilon\sigma\sqrt{\frac{k}{d}}}{\underline{\kappa}(1-\epsilon)}
  \end{equation}
  is sufficiently small, the linear program \eqref{eq:lp} recovers $R$, so that $\hat r_j = r_j$ with high probability.
  \end{thm}

The condition is essentially that 
  \begin{equation}
    \frac{1}{1-\epsilon} \sqrt{\frac{k}{d}}
  \end{equation}
  is small, meaning that the dimension $d$ of the symbols needs to be sufficiently large relative 
  to the dimension $k$ of the relation.

  \section{Sparse, high-dimensional relations} 

 The above setting ensures enough redundancy to recover the relations, constraining the number of symbols $k$ to be small relative to the symbol dimension $d$. This is not appropriate in the situation where the relations are over a large number $m$ of elements (for example the contents of the entire episodic memory).

 In this setting we assume that the relation tensor $R \in \reals^{m\times m}$ is sparse; that is, 
 each column $r_j \in \Delta_m$ has at most $k$ nonzero entries: $\|r_j\|_0 \leq k$. To recover the relation 
 we now use the robust lasso estimator (another linear program)
\begin{equation}
  \hat r_j = \argmin_{u \in\reals^k} \| a_j - S u\|_1 + \lambda\|u\|_1. \label{eq:rlasso}
\end{equation}  
Here we have an analogous theorem:
\begin{thm}
  Assume the symbol matrix $S$ satisfies Condition A and Condition B, and some other assumptions hold (not detailed). Suppose that 
\begin{eqnarray}
  \frac{\overline{\kappa}/\underline{\kappa}}{1-\epsilon}\sqrt{\frac{k}{d}\log(2m)}\leq c,
\end{eqnarray}
for some sufficiently small constant $c>0$. Then the robust lasso estimator \eqref{eq:rlasso} satisfies
\begin{equation}
  \|\hat r_j - r_j\| \leq C \frac{\overline{\kappa}/\underline{\kappa}^2}{1-\epsilon} \sqrt{\frac{\sigma^2 k}{d} \log(2m)}
\end{equation}
for some constant $C$.
\end{thm}
This implies that we can accurately recover the relation tensor in the high dimensional setting, even when many of the entries of the transformed abstract symbols are corrupted.

\section{Prediction without recovering the relations}

The above discussion shows how the relation tensor can be recovered from the transformed symbols, even under adversarial noise, assuming there is sufficient redundancy in the symbols. We'd like to go further, and show that it is possible to predict as well from the transformed symbols as from the relations, without explicitly recovering the relations. 

A possible route to analyzing this is through complexity measures.  
Suppose that we have a class of binary functions $\F_R$ on relations. Estimating a function $f\in\F_R$ 
from examples is a relational learning task---we have examples $y_i = f(R_i) + \eta_i$ and want to estimate $f$. The Rademacher complexity or 
VC dimension of $\F_R$ characterizes the sample complexity of such tasks.

An abstractor 
maps the relations to abstract symbols; the corresponding class of binary functions is denoted $\mathfrak{F}_A$. If we can 
show that the complexity of $\mathfrak{F}_A$ is approximately equal to the complexity of $\mathfrak{F}_R$,  
  this would imply that the relational task can be performed equally well on the abstract symbols. 

Typically for robust regression, where noise is added to the covariates, accurate prediction requires $\epsilon\to 0$, so that the fraction of corrupted variables is small. Here we may have a setting where 
$\epsilon$ can remain constant and accurate prediction is still possible.





% \section{Code for experiments}
\label{sec:code}

The supplementary material includes code for the experiments reported in \Cref{sec:experiments} of the paper. This code includes Python (Jupyter) notebooks for individual experiments as well as scripts that were run on a GPU cluster to parallelize across multiple experimental designs. Here, we give additional details on each experiment.

\subsection{Pairwise order relation}
The set up of the experiment is fully described in~\Cref{sec:experiments}. Here, we give details on the architecture we used.

The Abstractor model uses the implementation in \texttt{abstractor.py} which closely follows~\Cref{alg:abstractor}. The object pair are transformed by a linear embedder of dimension $64$ independently, then passed to an Abstractor module. The abstract objects output by the Abstractor are flattened to a single vector which is processed by a 32-unit dense layer with a relu activation. Finally, a 1-unit dense layer with sigmoid activation outputs the classification. The Abstractor module has 1 layer, a relation dimension of 4, a symbol dimension of 64, a projection dimension of 16, and a softmax relation activation function.

The Abstractor model is compared to CoRelNet. The same 64-dimensional linear embedder is used. This is used to compute the similarity matrix which is then flattened to form a single vector. The same 32-unit hidden layer and 1-unit classification layer are used.

We compute learning curves for each model by varying the training set size and evaluating the hold-out test set.

\subsection{Sorting random objects}

The dataset is described in~\Cref{sec:experiments}. We compare an Abstractor model, a Transformer model, and an Ablation model. The Abstractor model is of the form in~\Cref{alg:relational_abstractor}. The full autoregressive model uses the architecture $\module{Encoder} \to \module{Abstractor} \to \module{Decoder}$. For each of the Encoder, Abstractor, and Decoder modules, we use 2 layers, 2 attention heads/relation dimensions, a feedforward network with 64 units and an embedding dimension of 64. The number of trainable parameters is $386,954$. The Ablation model uses an identical architecture except that the relational attention is replaced with standard cross attention at the Encoder-Abstractor interface (with $Q \gets A, K \gets E, V \gets E$). It has the same number of parameters.

The Transformer model is standard and uses the $\module{Encoder} \to \module{Decoder}$ architecture. For both the Encoder and Decoder modules, we use 4 layers, 2 attention heads/relation dimensions, a feedforward network with 64 units and an embedding dimension of 64. The number of trainable parameters is $469,898$. We increased the number of layers compared to the Abstractor in order to make it a comparable size in terms of parameter count. Previously, we experimented with identical hyperparameters (where the Transformer would have fewer parameters due to not having an Abstractor module).

\subsection{Generalization to new object-sorting tasks}

\subsection{Robustness to noise and out-of-distribution generalization}

% \setcounter{subsection}{4}
\subsection{SET experiments}

This is a standalone implementation of an Abstractor that learns to classify triples of images of cards according 
to whether or not they form a SET, in an end-to-end fashion. First a convolutional neural network is trained to process the color images of the cards. The images are $70 \times 40$ pixels in size with $4$ color channels. The CNN has two convolutional layers, each with $32$ filters of size $5\times 5$ coupled with a $4\times 4$ max pooling layer. The feature maps are flattened and passed through two dense feedforward layers. The CNN is trained to predict the attribute of each card (one, two three; red, green, purple; empty, solid, striped; oval, diamond, squiggle), as a multi-label classification, and then an embedding of dimension $d=32$ of each card is obtained from the intermediate dense layer. 

Next, Abstractors are trained separately for each of the four attributes (number, color, pattern, shape) to learn same/different relations, where the task is to decide if an input pair of cards is the same or different for that attribute. The cards are encoded using the feature map generated from the CNN. We then use the query and key mappings $W_Q$ and $W_K$ learned for these relations to initialize the relations in a multi-head Abstractor. This Abstractor is then trained on a dataset of triples of cards, half of which form a SET, again representing the cards using the CNN feature maps.

The Abstractor is compared to a baseline model where the attributes and the relations are hand-coded symbolically, the relations are represented as vectors of 12 bits. A two-layer MLP is then trained to decide if the triple forms a SET. The MLP using the symbolic representation can be considered as a lower bound on the performance achievable by the Abstractor. We note that the simple Abstractor makes no use of self-attention or any of the other enhancements such as skip-connections commonly used in Transformers.


\end{document}